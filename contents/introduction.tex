\chapter{引言}

\section{研究背景}

放射治疗(radiation therapy,简称放疗)是现代肿瘤综合治疗体系中的重要组成部分,其基本思想是利用直线加速器或放射性核素产生的高能电离辐射,对恶性细胞造成不可逆损伤,从而实现肿瘤局部控制甚至根治。对于局限于特定解剖区域、尚未出现远处转移的实体瘤,例如部分早期乳腺癌、头颈部肿瘤等,单纯放疗或术后辅助放疗往往可以显著降低复发风险并改善长期生存率。在临床实践中,放疗常与外科手术、化学治疗、内分泌治疗及免疫治疗等手段协同使用,根据肿瘤类型和分期的不同,可在手术前、术中或术后,以及化疗前后灵活组合。围绕放射治疗开展的专科被称为放射肿瘤学,其核心工作包括处方剂量的确定、照射方案设计以及治疗实施与评估。

放疗之所以能够选择性地抑制肿瘤生长,是因为电离辐射在组织中沉积能量最后损伤 DNA 及相关微观结构,使肿瘤细胞的增殖能力受到破坏。然而,射线在进入人体后不可避免地穿过皮肤、正常器官及其他危及器官,因此临床放疗计划的一个基本目标,是在保证靶区获得足够剂量的同时,尽可能降低周围健康组织的受量。为此,治疗计划系统通常会利用多个入射方向、不同能量和强度调制的射线束,通过几何构型和优化算法,使得各束在肿瘤区域“叠加”成较高的吸收剂量,而在周边组织中形成相对较低的剂量分布。此外,若引流淋巴结在影像学或临床评估中提示存在受累或存在亚临床播散的风险,相应区域也会被纳入放疗靶区。由于患者体位摆放误差、呼吸运动、膀胱充盈等因素会导致靶区在治疗过程中发生位置和形态变化,临床上通常需要在名义靶区外加入一定安全边界,以提高实际照射过程中对肿瘤体积的覆盖概率。

在这一背景下,高精度的剂量计算方法就成为放疗计划设计的关键环节。传统上,临床治疗计划系统多采用基于经验的剂量卷积–叠加模型或简化的能量分布近似,这类方法在均匀介质和简单几何条件下表现尚可,但在存在组织非均匀性(如肺、骨、空腔)或复杂边界条件时,往往难以准确描述粒子在组织中的多次散射、能量损失和角度扩展等微观输运过程。尤其是在肺癌、头颈部肿瘤以及质子、重离子等粒子束放疗场景下,介质密度梯度明显、散射机制复杂,简单近似模型的误差可能显著放大,从而引发剂量评估的不确定性。

从理论角度来看,无论是光子束、电子束,还是质子及重离子束,其在介质中的传播本质上都可以用辐射输运方程(radiative transfer equation, RTE)或线性玻尔兹曼输运方程(linear Boltzmann transport equation)加以刻画。该类输运方程以相空间(空间位置–速度方向–能量)为自变量,综合描述粒子的散射、吸收和源项等物理过程,因此能够在统一框架下给出更为精细的剂量沉积分布。然而,直接求解高维输运方程的数值代价巨大,既要处理复杂的边界和源条件(例如多方向外照射束、内部放射源等),又要兼顾临床应用对计算效率和稳定性的严格要求。与此相对,蒙特卡洛方法虽然在理论上可以逼近真实输运过程,被视为剂量计算的“金标准”,但在常规临床工作流程中,其计算成本仍然是一个重要限制因素。

因此,如何在较低的计算代价下,既保留输运方程的物理精度,又兼顾临床可用性,成为当前放疗剂量计算领域的重要研究课题。一方面,基于确定性数值方法的线性玻尔兹曼方程求解器(如若干商用系统中的线性输运引擎)已经在部分临床应用中显示出良好前景;另一方面,近年来兴起的神经算子(neural operator)、深度学习加速输运求解等方法,为在保持物理一致性的前提下,通过学习复杂介质和束流条件下的解算算子,提供了新的思路。特别是在粒子束放疗(如质子束和重离子束)中,束流能谱、非弹性散射和多通道能量沉积分布更加复杂,如何基于辐射输运方程构建可泛化的求解框架,并进一步与数据驱动的模型相结合,是本文后续工作的主要动机之一。


\section{辐射输运问题}

从前一章可以看出,在放射治疗以及更广泛的粒子束输运问题中,高能光子、电子以及质子和重离子等带电粒子在组织或一般介质中的传播规律,是决定剂量沉积分布和物理响应的核心因素。为了在统一的数学框架下刻画这些不同类型粒子的传播、散射与吸收行为,本章将从辐射输运方程(radiative transfer equation, RTE)与线性玻尔兹曼输运方程(linear Boltzmann transport equation, LBTE)出发,引入相空间的概念并系统讨论相关建模问题。

% 本章关注的是:如何在相空间中对粒子分布进行描述,以及怎样通过输运方程在理论上统一处理光子束、电子束和各种粒子束(例如质子束、重离子束)在复杂介质中的传播。具体而言,RTE/LBTE 以空间位置、传播方向乃至能量变量构成的相空间作为自变量,将散射、吸收和外部源项等物理机制编码到一个偏微分–积分方程中,从而为后续的数值离散、算法设计以及基于神经算子的求解框架提供严格的模型基础和统一的语言。

\subsection{玻尔兹曼方程简介}

动理学理论中,玻尔兹曼方程(Boltzmann equation),亦称玻尔兹曼输运方程(Boltzmann transport equation, BTE),用于刻画远离热力学平衡状态的多粒子体系的统计行为。该方程最早由 Ludwig Boltzmann 于 19 世纪提出,经典例子是存在空间温度梯度的流体:由于高温区域和低温区域之间存在温差,微观粒子的随机运动在统计意义上呈现出“有偏”的输运,从而形成宏观可观测的热流。在当代文献中,“玻尔兹曼方程”这一术语往往被广义使用,泛指一大类动理学方程,用来刻画热力学系统中某种宏观量(如能量、粒子数、电荷等)在非平衡过程中的演化。

玻尔兹曼方程并不直接追踪系统中每一个粒子的具体位置和动量轨迹,而是引入相空间中的分布函数这一概念。具体来说,玻尔兹曼方程描述的是粒子在$t$时刻,相空间$(\br, \bv)$处的概率密度函数随时间、空间和动量变量变化的偏微分–积分方程。

一般的,玻尔兹曼方程可以写做以下格式:
\begin{equation}
    \partial_t f + \bv \cdot \nabla_{\br} f = \mathcal{C}(f), \quad t>0,\quad (\br, \bv) \in \mathbb{R}^{d_x} \times \mathbb{R}^{d_v},
\end{equation}
其中,$f = f(t, \br, \bv)$为分布函数,表示在时刻$t$,位置$\br$处,动量方向为$\bv$的粒子数密度;$\mathcal{C}$为碰撞算子,描述粒子在相空间中由于散射、吸收等过程引起的分布函数变化。 碰撞算子的具体形式依赖于所研究的物理系统及其相互作用机制。例如,在气体动力学中,碰撞算子通常采用Boltzmann碰撞积分形式,考虑粒子间的二体碰撞;而在辐射输运中,碰撞算子则包含吸收和散射项,反映光子与介质的相互作用。

在经典动理学理论框架下,玻尔兹曼方程不仅给出非平衡分布函数 $f(t,\br,\bv)$ 的演化方程,还蕴含了一系列与热力学和流体力学密切相关的结构性性质。

\paragraph{1.\ 平衡态与麦克斯韦分布}

玻尔兹曼方程的平衡态(或局部平衡态)通常由麦克斯韦分布(Maxwellian)给出。设平衡态分布记为 $f_M$,若在该态下碰撞算子满足
\begin{equation}
    \mathcal{C}(f_M) = 0,
\end{equation}
则称 $f_M$ 为碰撞平衡态。其典型形式为
\begin{equation}
    f_M(t,\br,\bv)
    = \frac{\rho(t,\br)}{m} \, \frac{1}{\bigl(2\pi \theta(t,\br)\bigr)^{3/2}}
    \exp\!\left(-\frac{\lvert \bv - \bm{u}(t,\br)\rvert^2}{2\,\theta(t,\br)}\right),
\end{equation}
其中 $\rho(t,\br)$ 为质量密度,$m$ 为单个粒子的质量,$\theta(t,\br)$ 与温度成正比,$\bm{u}(t,\br)$ 为宏观速度场。若真实分布 $f$ 偏离局部麦克斯韦分布 $f_M$,碰撞作用会在时间演化中驱动 $f$ 向 $f_M$ 逐渐靠拢,因此 $f_M$ 在适当意义下是系统的吸引态。

从速度矩(velocity moments)的角度来看,上述宏观量都可以视为对分布函数 $f$ 的不同阶矩。首先,质量密度 $\rho(t,\br)$ 是分布函数关于速度变量的零阶矩:
\begin{equation}
    \rho(t,\br)
    = m \int_{\mathbb{R}^{d_v}} f(t,\br,\bv)\,\mathrm{d}\bv,
\end{equation}
其中 $m$ 为单个粒子的质量。其次,宏观速度场 $\bm{u}(t,\br)$ 来自分布函数的一阶矩,满足
\begin{equation}
    \rho(t,\br)\,\bm{u}(t,\br)
    = m \int_{\mathbb{R}^{d_v}} \bv\,f(t,\br,\bv)\,\mathrm{d}\bv,
\end{equation}
即 $\bm{u}(t,\br)$ 可以理解为以 $f$ 为权函数的速度平均值。进一步地,与温度成正比的量 $\theta(t,\br)$ 可由二阶矩给出:
\begin{equation}
    \frac{d_v}{2}\,\rho(t,\br)\,\theta(t,\br)
    = \frac{m}{2} \int_{\mathbb{R}^{d_v}} \lvert \bv - \bm{u}(t,\br)\rvert^2
    f(t,\br,\bv)\,\mathrm{d}\bv,
\end{equation}
右端可视为单位体积内的平均动能密度。给定上述由零阶、一阶和二阶速度矩确定的宏观量,在所有满足相应矩约束的分布函数中,麦克斯韦分布 $f_M$ 是使熵(或 $H$ 泛函)达到极大值的一类分布,因此对应于玻尔兹曼方程的平衡态。

\paragraph{2.\ 守恒定律}

玻尔兹曼方程的碰撞算子反映了粒子间的微观相互作用。对于质量、动量和能量这样的碰撞不变量,碰撞过程本身不改变其总量,这可表述为以下积分恒等式:
\begin{equation}
    \begin{aligned}
        \text{质量守恒:}\quad
         & m \int_{\mathbb{R}^{d_v}} \mathcal{C}(f)\,\mathrm{d}\bv = 0,                              \\
        \text{动量守恒:}\quad
         & m \int_{\mathbb{R}^{d_v}} \bv\,\mathcal{C}(f)\,\mathrm{d}\bv = \bm{0},                    \\
        \text{能量守恒:}\quad
         & \frac{m}{2} \int_{\mathbb{R}^{d_v}} \lvert \bv\rvert^2 \mathcal{C}(f)\,\mathrm{d}\bv = 0.
    \end{aligned}
\end{equation}
对玻尔兹曼方程分别以 $1$、$\bv$ 和 $\lvert\bv\rvert^2$ 作为权函数取速度矩,可以得到对应的质量守恒方程(连续性方程)、动量平衡方程和能量平衡方程,从而在宏观尺度上导出流体力学的基本控制方程(如欧拉方程或 Navier--Stokes 方程)。

\paragraph{3.\ H 定理与熵增性质}

玻尔兹曼在其理论中引入了熵密度泛函
\begin{equation}
    \eta(t,\br) = -k \int_{\mathbb{R}^{d_v}} f(t,\br,\bv)\,
    \ln\!\frac{f(t,\br,\bv)}{y}\,\mathrm{d}\bv,
\end{equation}
其中 $k$ 为玻尔兹曼常数,$y$ 为一常数(例如与归一化有关)。在孤立体系且无外源的情形下,H 定理表明碰撞过程导致的熵产生率满足
\begin{equation}
    \Sigma(t,\br) \ge 0,
\end{equation}
即体系总熵随时间演化单调不减,只有当 $f$ 达到麦克斯韦平衡态时熵产生率才为零,此时熵取得极大值。该结论与热力学第二定律具有一致性,从动力学角度揭示了非平衡系统向平衡态演化的不可逆性,也为后续讨论“从动理学方程到宏观不可逆输运过程”的分析提供了数学基础。

直接求玻尔兹曼方程的解析解只在极少数理想化情形下可行,因此实际应用中通常依赖数值近似。现有数值方法大致可以分为以下几类:一类是基于相空间离散的确定性方法,包括直接离散玻尔兹曼(discrete velocity method)、谱方法、有限体积或有限元方法等,通过在速度空间选取有限个代表速度或基函数,将碰撞积分做一个近似,从而得到可求解的方程组;一类是格子玻尔兹曼方法(lattice Boltzmann method),在规则的空间–速度格点上定义简化的碰撞–输运规则,用于模拟近似不可压流体或多相流等宏观行为;另一大类是蒙特卡洛型随机方法,如直模拟蒙特卡洛(direct simulation Monte Carlo, DSMC)及其变体,通过随机采样粒子和碰撞过程来统计分布函数或宏观量的演化,这类方法在处理高维相空间时具有一定灵活性,但噪声和收敛效率是重要考量。

本文主要关心的是玻尔兹曼方程在合适物理假设下得到的约化模型,尤其是其中的线性玻尔兹曼输运方程及其在辐射输运问题中的具体形式。

\subsection{辐射输运方程简介}

在数学物理中,线性输运理论研究的是一类描述粒子或能量在介质中迁移的方程,其迁移过程伴随着随机的吸收、发射和散射事件;在玻尔兹曼方程框架下,当碰撞算子右端可以用分布函数的线性算子来近似时,便得到线性输运方程,而辐射输运方程(radiative transfer equation, RTE)正是这类线性输运方程在光学和辐射传输问题中的典型代表。RTE 广泛用于刻画中子输运、天体和大气辐射传输、参量热辐射传热以及光学成像等问题中粒子或光子的传播与相互作用过程。其核心思想是:在给定介质中,高能粒子(或光子)在空间–角度相空间中随路径前进,同时经历吸收、发射和散射等微观过程,由此形成宏观上可观测的辐射强度和能量沉积分布。

从最一般的形式来看,辐射输运方程是关于时间 $t$、空间位置 $\br$、传播方向 $\bOmega$ 和能量(或频率)$E$ 的函数 $I(t,\br,\bOmega,E)$ 的控制方程,其中 $I$ 表示在 $\br$ 处、沿方向 $\bOmega$、能量在 $[E,E+\mathrm{d}E)$ 内的粒子通量密度。考虑同时存在外源、散射产生粒子以及(在中子输运等情形下的)裂变源的影响下,一般的输运方程可以写成
\begin{equation}
    \begin{aligned}
        \frac{1}{v(E)} \, \partial_t I(t,\br,\bOmega,E)
         & + \bOmega \cdot \nabla_{\br} I(t,\br,\bOmega,E) + \Sigma_t(\br,E) \, I(t,\br,\bOmega,E) \\
         & = q_{\mathrm{ex}}(t,\br,\bOmega,E) + q_s(t,\br,\bOmega,E) + q_f(t,\br,\bOmega,E),
    \end{aligned}
\end{equation}
其中 $v(E)$ 为能量为 $E$ 的粒子速度,$\Sigma_t(\br,E)$ 为总宏观截面(total cross section),综合反映在位置 $\br$、能量 $E$ 处单位长度内所有相互作用(吸收、散射、裂变等)发生的概率;$q_{\mathrm{ex}}$ 为给定的外源项,表示与通量分布 $I$ 无关的入射粒子源;$q_s$ 和 $q_f$ 分别为散射源和裂变源,它们都依赖于当前的粒子分布 $I$。在无裂变的非乘倍系统中有 $q_f \equiv 0$,只需考虑外源和散射源两部分贡献。

以非乘倍系统为例($q_f=0$),散射源 $q_s$ 可以通过对入射能量和方向积分得到:
\begin{equation}
    q_s(t,\br,\bOmega,E)
    = \int_0^{\infty} \int_{\sS^{d-1}}
    \Sigma_s\bigl(\br; E' \to E,\bOmega' \cdot \bOmega\bigr) \,
    I(t,\br,\bOmega',E')\,\mathrm{d}\bOmega'\,\mathrm{d}E',
\end{equation}
其中 $\Sigma_s(\br; E' \to E,\bOmega' \cdot \bOmega)$ 为散射截面核,描述在位置 $\br$,能量为 $E'$、沿方向 $\bOmega'$ 的粒子在单位路径长度内被散射到能量 $E$、方向 $\bOmega$ 的概率密度;对所有可能的入射能量 $E'$ 和方向 $\bOmega'$ 积分即可得到单位体积、单位时间内由散射产生的二次粒子源强。类似地,在存在可裂变核素的乘倍系统中,还需引入由裂变过程贡献的源项 $q_f$,它通常由裂变截面与裂变中子(或其它次级粒子)能谱叠加得到。



在许多辐射输运与医学物理问题中,若过程可视为稳态($\partial_t I = 0$)、并且能量变量经过适当简化,则上述一般输运方程可以退化为只含位置与角度的稳态 RTE。为避免符号混乱,本文在后续章节中主要采用简化形式讨论 RTE 的解算算子结构和数值性质,但在概念上始终可以从上述时空–能量–角度统一的高维输运方程出发来理解各类近似模型。

与前文讨论的玻尔兹曼方程类似,RTE 也是定义在高维相空间的偏微分–积分方程。在三维空间、二维角度变量($d=3$)的典型情形下,自变量维数达到七维,且散射积分项在角度空间引入了全局耦合,使得解析解一般不可得,只能依赖数值方法。现有 RTE 数值方法可以粗略分为两大类。一类是基于偏微分方程求解器的确定性方法:在空间上常采用菱形差分(diamond difference)、迎风格式和其他有限差分方法,或有限元、有限体积、间断 Galerkin 等离散手段;在角度上则使用 $P_n$ 方法、有限元或离散纵标(discrete ordinates, DOM)等,将角度积分用有限和近似。离散之后需要求解规模较大的线性代数系统,且由于散射积分的存在,系数矩阵往往并不十分稀疏,因而通常需要迭代法并配合加速技术。另一类是蒙特卡洛方法,它通过在相空间中随机采样粒子轨迹和散射事件,统计近似 RTE 解,具有几何处理灵活、易于并行等优点,但收敛速度遵循 $1/\sqrt{N}$ 规律,高精度计算往往需要大量样本,计算代价较高。