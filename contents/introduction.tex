\chapter{引言}

\section{研究背景}

放射治疗(radiation therapy,简称放疗)是现代肿瘤综合治疗体系中的重要组成部分,其基本思想是利用直线加速器或放射性核素产生的高能电离辐射,对恶性细胞造成不可逆损伤,从而实现肿瘤局部控制甚至根治。对于局限于特定解剖区域、尚未出现远处转移的实体瘤,例如部分早期乳腺癌、头颈部肿瘤等,单纯放疗或术后辅助放疗往往可以显著降低复发风险并改善长期生存率。在临床实践中,放疗常与外科手术、化学治疗、内分泌治疗及免疫治疗等手段协同使用,根据肿瘤类型和分期的不同,可在手术前、术中或术后,以及化疗前后灵活组合。围绕放射治疗开展的专科被称为放射肿瘤学,其核心工作包括处方剂量的确定、照射方案设计以及治疗实施与评估。

放疗之所以能够选择性地抑制肿瘤生长,是因为电离辐射在组织中沉积能量最后损伤 DNA 及相关微观结构,使肿瘤细胞的增殖能力受到破坏。然而,射线在进入人体后不可避免地穿过皮肤、正常器官及其他危及器官,因此临床放疗计划的一个基本目标,是在保证靶区获得足够剂量的同时,尽可能降低周围健康组织的受量。为此,治疗计划系统通常会利用多个入射方向、不同能量和强度调制的射线束,通过几何构型和优化算法,使得各束在肿瘤区域“叠加”成较高的吸收剂量,而在周边组织中形成相对较低的剂量分布。此外,若引流淋巴结在影像学或临床评估中提示存在受累或存在亚临床播散的风险,相应区域也会被纳入放疗靶区。由于患者体位摆放误差、呼吸运动、膀胱充盈等因素会导致靶区在治疗过程中发生位置和形态变化,临床上通常需要在名义靶区外加入一定安全边界,以提高实际照射过程中对肿瘤体积的覆盖概率。

在这一背景下,高精度的剂量计算方法就成为放疗计划设计的关键环节。传统上,临床治疗计划系统多采用基于经验的剂量卷积–叠加模型或简化的能量分布近似,这类方法在均匀介质和简单几何条件下表现尚可,但在存在组织非均匀性(如肺、骨、空腔)或复杂边界条件时,往往难以准确描述粒子在组织中的多次散射、能量损失和角度扩展等微观输运过程。尤其是在肺癌、头颈部肿瘤以及质子、重离子等粒子束放疗场景下,介质密度梯度明显、散射机制复杂,简单近似模型的误差可能显著放大,从而引发剂量评估的不确定性。

从理论角度来看,无论是光子束、电子束,还是质子及重离子束,其在介质中的传播本质上都可以用辐射输运方程(radiative transfer equation, RTE)或线性玻尔兹曼输运方程(linear Boltzmann transport equation)加以刻画。该类输运方程以相空间(空间位置–速度方向–能量)为自变量,综合描述粒子的散射、吸收和源项等物理过程,因此能够在统一框架下给出更为精细的剂量沉积分布。然而,直接求解高维输运方程的数值代价巨大,既要处理复杂的边界和源条件(例如多方向外照射束、内部放射源等),又要兼顾临床应用对计算效率和稳定性的严格要求。与此相对,蒙特卡洛方法虽然在理论上可以逼近真实输运过程,被视为剂量计算的“金标准”,但在常规临床工作流程中,其计算成本仍然是一个重要限制因素。

因此,如何在较低的计算代价下,既保留输运方程的物理精度,又兼顾临床可用性,成为当前放疗剂量计算领域的重要研究课题。一方面,基于确定性数值方法的线性玻尔兹曼方程求解器(如若干商用系统中的线性输运引擎)已经在部分临床应用中显示出良好前景;另一方面,近年来兴起的神经算子(neural operator)、深度学习加速输运求解等方法,为在保持物理一致性的前提下,通过学习复杂介质和束流条件下的解算算子,提供了新的思路。特别是在粒子束放疗(如质子束和重离子束)中,束流能谱、非弹性散射和多通道能量沉积分布更加复杂,如何基于辐射输运方程构建可泛化的求解框架,并进一步与数据驱动的模型相结合,是本文后续工作的主要动机之一。


\section{辐射输运问题}

从前一章可以看出,在放射治疗以及更广泛的粒子束输运问题中,高能光子、电子以及质子和重离子等带电粒子在组织或一般介质中的传播规律,是决定剂量沉积分布和物理响应的核心因素。为了在统一的数学框架下刻画这些不同类型粒子的传播、散射与吸收行为,本章将从辐射输运方程(radiative transfer equation, RTE)与线性玻尔兹曼输运方程(linear Boltzmann transport equation, LBTE)出发,引入相空间的概念并系统讨论相关建模问题。

% 本章关注的是:如何在相空间中对粒子分布进行描述,以及怎样通过输运方程在理论上统一处理光子束、电子束和各种粒子束(例如质子束、重离子束)在复杂介质中的传播。具体而言,RTE/LBTE 以空间位置、传播方向乃至能量变量构成的相空间作为自变量,将散射、吸收和外部源项等物理机制编码到一个偏微分–积分方程中,从而为后续的数值离散、算法设计以及基于神经算子的求解框架提供严格的模型基础和统一的语言。

\subsection{玻尔兹曼方程简介}

动理学理论中,玻尔兹曼方程(Boltzmann equation),亦称玻尔兹曼输运方程(Boltzmann transport equation, BTE),用于刻画远离热力学平衡状态的多粒子体系的统计行为。该方程最早由 Ludwig Boltzmann 于 19 世纪提出,经典例子是存在空间温度梯度的流体:由于高温区域和低温区域之间存在温差,微观粒子的随机运动在统计意义上呈现出“有偏”的输运,从而形成宏观可观测的热流。在当代文献中,“玻尔兹曼方程”这一术语往往被广义使用,泛指一大类动理学方程,用来刻画热力学系统中某种宏观量(如能量、粒子数、电荷等)在非平衡过程中的演化。

玻尔兹曼方程并不直接追踪系统中每一个粒子的具体位置和动量轨迹,而是引入相空间中的分布函数这一概念。具体来说,玻尔兹曼方程描述的是粒子在$t$时刻,相空间$(\br, \bv)$处的概率密度函数随时间、空间和动量变量变化的偏微分–积分方程。

一般的,玻尔兹曼方程可以写做以下格式:
\begin{equation}
    \partial_t f + \bv \cdot \nabla_{\br} f = \mathcal{C}(f), \quad t>0,\quad (\br, \bv) \in \mathbb{R}^{d_x} \times \mathbb{R}^{d_v},
\end{equation}
其中,$f = f(t, \br, \bv)$为分布函数,表示在时刻$t$,位置$\br$处,动量方向为$\bv$的粒子数密度;$\mathcal{C}$为碰撞算子,描述粒子在相空间中由于散射、吸收等过程引起的分布函数变化。 碰撞算子的具体形式依赖于所研究的物理系统及其相互作用机制。例如,在气体动力学中,碰撞算子通常采用Boltzmann碰撞积分形式,考虑粒子间的二体碰撞;而在辐射输运中,碰撞算子则包含吸收和散射项,反映光子与介质的相互作用。

在经典动理学理论框架下,玻尔兹曼方程不仅给出非平衡分布函数 $f(t,\br,\bv)$ 的演化方程,还蕴含了一系列与热力学和流体力学密切相关的结构性性质。

\paragraph{1.\ 平衡态与麦克斯韦分布}

玻尔兹曼方程的平衡态(或局部平衡态)通常由麦克斯韦分布(Maxwellian)给出。设平衡态分布记为 $f_M$,若在该态下碰撞算子满足
\begin{equation}
    \mathcal{C}(f_M) = 0,
\end{equation}
则称 $f_M$ 为碰撞平衡态。其典型形式为
\begin{equation}
    f_M(t,\br,\bv)
    = \frac{\rho(t,\br)}{m} \, \frac{1}{\bigl(2\pi \theta(t,\br)\bigr)^{3/2}}
    \exp\!\left(-\frac{\lvert \bv - \bm{u}(t,\br)\rvert^2}{2\,\theta(t,\br)}\right),
\end{equation}
其中 $\rho(t,\br)$ 为质量密度,$m$ 为单个粒子的质量,$\theta(t,\br)$ 与温度成正比,$\bm{u}(t,\br)$ 为宏观速度场。若真实分布 $f$ 偏离局部麦克斯韦分布 $f_M$,碰撞作用会在时间演化中驱动 $f$ 向 $f_M$ 逐渐靠拢,因此 $f_M$ 在适当意义下是系统的吸引态。

从速度矩(velocity moments)的角度来看,上述宏观量都可以视为对分布函数 $f$ 的不同阶矩。首先,质量密度 $\rho(t,\br)$ 是分布函数关于速度变量的零阶矩:
\begin{equation}
    \rho(t,\br)
    = m \int_{\mathbb{R}^{d_v}} f(t,\br,\bv)\,\mathrm{d}\bv,
\end{equation}
其中 $m$ 为单个粒子的质量。其次,宏观速度场 $\bm{u}(t,\br)$ 来自分布函数的一阶矩,满足
\begin{equation}
    \rho(t,\br)\,\bm{u}(t,\br)
    = m \int_{\mathbb{R}^{d_v}} \bv\,f(t,\br,\bv)\,\mathrm{d}\bv,
\end{equation}
即 $\bm{u}(t,\br)$ 可以理解为以 $f$ 为权函数的速度平均值。进一步地,与温度成正比的量 $\theta(t,\br)$ 可由二阶矩给出:
\begin{equation}
    \frac{d_v}{2}\,\rho(t,\br)\,\theta(t,\br)
    = \frac{m}{2} \int_{\mathbb{R}^{d_v}} \lvert \bv - \bm{u}(t,\br)\rvert^2
    f(t,\br,\bv)\,\mathrm{d}\bv,
\end{equation}
右端可视为单位体积内的平均动能密度。给定上述由零阶、一阶和二阶速度矩确定的宏观量,在所有满足相应矩约束的分布函数中,麦克斯韦分布 $f_M$ 是使熵(或 $H$ 泛函)达到极大值的一类分布,因此对应于玻尔兹曼方程的平衡态。

\paragraph{2.\ 守恒定律}

玻尔兹曼方程的碰撞算子反映了粒子间的微观相互作用。对于质量、动量和能量这样的碰撞不变量,碰撞过程本身不改变其总量,这可表述为以下积分恒等式:
\begin{equation}
    \begin{aligned}
        \text{质量守恒:}\quad
         & m \int_{\mathbb{R}^{d_v}} \mathcal{C}(f)\,\mathrm{d}\bv = 0,                              \\
        \text{动量守恒:}\quad
         & m \int_{\mathbb{R}^{d_v}} \bv\,\mathcal{C}(f)\,\mathrm{d}\bv = \bm{0},                    \\
        \text{能量守恒:}\quad
         & \frac{m}{2} \int_{\mathbb{R}^{d_v}} \lvert \bv\rvert^2 \mathcal{C}(f)\,\mathrm{d}\bv = 0.
    \end{aligned}
\end{equation}
对玻尔兹曼方程分别以 $1$、$\bv$ 和 $\lvert\bv\rvert^2$ 作为权函数取速度矩,可以得到对应的质量守恒方程(连续性方程)、动量平衡方程和能量平衡方程,从而在宏观尺度上导出流体力学的基本控制方程(如欧拉方程或 Navier--Stokes 方程)。

\paragraph{3.\ H 定理与熵增性质}

玻尔兹曼在其理论中引入了熵密度泛函
\begin{equation}
    \eta(t,\br) = -k \int_{\mathbb{R}^{d_v}} f(t,\br,\bv)\,
    \ln\!\frac{f(t,\br,\bv)}{y}\,\mathrm{d}\bv,
\end{equation}
其中 $k$ 为玻尔兹曼常数,$y$ 为一常数(例如与归一化有关)。在孤立体系且无外源的情形下,H 定理表明碰撞过程导致的熵产生率满足
\begin{equation}
    \Sigma(t,\br) \ge 0,
\end{equation}
即体系总熵随时间演化单调不减,只有当 $f$ 达到麦克斯韦平衡态时熵产生率才为零,此时熵取得极大值。该结论与热力学第二定律具有一致性,从动力学角度揭示了非平衡系统向平衡态演化的不可逆性,也为后续讨论“从动理学方程到宏观不可逆输运过程”的分析提供了数学基础。

直接求玻尔兹曼方程的解析解只在极少数理想化情形下可行,因此实际应用中通常依赖数值近似。现有数值方法大致可以分为以下几类:一类是基于相空间离散的确定性方法,包括直接离散玻尔兹曼(discrete velocity method)、谱方法、有限体积或有限元方法等,通过在速度空间选取有限个代表速度或基函数,将碰撞积分做一个近似,从而得到可求解的方程组;一类是格子玻尔兹曼方法(lattice Boltzmann method),在规则的空间–速度格点上定义简化的碰撞–输运规则,用于模拟近似不可压流体或多相流等宏观行为;另一大类是蒙特卡洛型随机方法,如直模拟蒙特卡洛(direct simulation Monte Carlo, DSMC)及其变体,通过随机采样粒子和碰撞过程来统计分布函数或宏观量的演化,这类方法在处理高维相空间时具有一定灵活性,但噪声和收敛效率是重要考量。

本文主要关心的是玻尔兹曼方程在合适物理假设下得到的约化模型,尤其是其中的线性玻尔兹曼输运方程及其在辐射输运问题中的具体形式。

\subsection{辐射输运方程}

在数学物理中,线性输运理论研究的是一类描述粒子或能量在介质中迁移的方程,其迁移过程伴随着随机的吸收、发射和散射事件;在玻尔兹曼方程框架下,当碰撞算子右端可以用分布函数的线性算子来近似时,便得到线性输运方程,而辐射输运方程(radiative transfer equation, RTE)正是这类线性输运方程在光学和辐射传输问题中的典型代表。RTE 广泛用于刻画中子输运、天体和大气辐射传输、参量热辐射传热以及光学成像等问题中粒子或光子的传播与相互作用过程。其核心思想是:在给定介质中,高能粒子(或光子)在空间–角度相空间中随路径前进,同时经历吸收、发射和散射等微观过程,由此形成宏观上可观测的辐射强度和能量沉积分布。

从最一般的形式来看,辐射输运方程是关于时间 $t$、空间位置 $\br$、传播方向 $\bOmega$ 和能量(或频率)$E$ 的函数 $I(t,\br,\bOmega,E)$ 的控制方程,其中 $I$ 表示在 $\br$ 处、沿方向 $\bOmega$、能量在 $[E,E+\mathrm{d}E)$ 内的粒子通量密度。考虑同时存在外源、散射产生粒子以及(在中子输运等情形下的)裂变源的影响下,一般的输运方程可以写成
\begin{equation}
    \begin{aligned}
        \frac{1}{v(E)} \, \partial_t I(t,\br,\bOmega,E)
         & + \bOmega \cdot \nabla_{\br} I(t,\br,\bOmega,E) + \Sigma_t(\br,E) \, I(t,\br,\bOmega,E) \\
         & = q_{\mathrm{ex}}(t,\br,\bOmega,E) + q_s(t,\br,\bOmega,E) + q_f(t,\br,\bOmega,E),
    \end{aligned}
\end{equation}
其中 $v(E)$ 为能量为 $E$ 的粒子速度,$\Sigma_t(\br,E)$ 为总宏观截面(total cross section),综合反映在位置 $\br$、能量 $E$ 处单位长度内所有相互作用(吸收、散射、裂变等)发生的概率;$q_{\mathrm{ex}}$ 为给定的外源项,表示与通量分布 $I$ 无关的入射粒子源;$q_s$ 和 $q_f$ 分别为散射源和裂变源,它们都依赖于当前的粒子分布 $I$。在无裂变的非乘倍系统中有 $q_f \equiv 0$,只需考虑外源和散射源两部分贡献。

以非乘倍系统为例($q_f=0$),散射源 $q_s$ 可以通过对入射能量和方向积分得到:
\begin{equation}
    q_s(t,\br,\bOmega,E)
    = \int_0^{\infty} \int_{\sS^{d-1}}
    \Sigma_s\bigl(\br; E' \to E,\bOmega' \cdot \bOmega\bigr) \,
    I(t,\br,\bOmega',E')\,\mathrm{d}\bOmega'\,\mathrm{d}E',
\end{equation}
其中 $\Sigma_s(\br; E' \to E,\bOmega' \cdot \bOmega)$ 为散射截面核,描述在位置 $\br$,能量为 $E'$、沿方向 $\bOmega'$ 的粒子在单位路径长度内被散射到能量 $E$、方向 $\bOmega$ 的概率密度;对所有可能的入射能量 $E'$ 和方向 $\bOmega'$ 积分即可得到单位体积、单位时间内由散射产生的二次粒子源强。类似地,在存在可裂变核素的乘倍系统中,还需引入由裂变过程贡献的源项 $q_f$,它通常由裂变截面与裂变中子(或其它次级粒子)能谱叠加得到。

\paragraph{散射截面与散射核}
在辐射输运问题中,散射截面刻画了粒子在介质中发生能量与方向重分布的统计规律。给定空间位置 $\br$ 及入射能量 $E$,总散射宏观截面 $\Sigma_s(\br,E)$ 表示粒子沿路径前进时单位长度内发生任意散射事件的平均概率强度;与之相配的散射分布函数
\begin{equation}
    f(E \to E', \bOmega \cdot \bOmega')
    = \frac{\Sigma_s(\br; E \to E', \bOmega \cdot \bOmega')}{\Sigma_s(\br,E)}
\end{equation}
描述在已发生一次散射的条件下,粒子从入射态 $(E,\bOmega)$ 被散射到出射态 $(E',\bOmega')$ 的能角分布。对于每次仅产生一个次级粒子的散射过程,$f$ 作为条件概率密度通常满足
\begin{equation}
    \int_0^{\infty}\!\int_{4\pi}
    f(E \to E', \bOmega \cdot \bOmega')\,
    \mathrm{d}\Omega'\,\mathrm{d}E' = 1,
\end{equation}
从而微分散射截面核可写为
\begin{equation}
    \Sigma_s(\br; E \to E', \bOmega \cdot \bOmega')
    = \Sigma_s(\br,E)\, f(E \to E', \bOmega \cdot \bOmega'),
\end{equation}
并满足对出射能量和方向积分后恢复总散射截面
\begin{equation}
    \int_0^{\infty}\!\int_{4\pi}
    \Sigma_s(\br; E \to E', \bOmega \cdot \bOmega')\,
    \mathrm{d}\Omega'\,\mathrm{d}E'
    = \Sigma_s(\br,E).
\end{equation}

在各向同性介质中,散射对方位角无偏好,散射核仅依赖于散射角余弦 $\mu = \bOmega \cdot \bOmega'$,故常将其记作 $f(E \to E', \mu)$,并可通过对 $\mu$ 的展开(如勒让德多项式展开)来表征角向各阶各向异性成分。在多通道散射的情形下,例如同时存在弹性和非弹性散射,可分别引入各自的散射截面和分布函数,并按截面大小对其加权,从而得到一个统一的“有效”散射核;该核在形式上仍满足与单通道情形相同的归一化性质与保号性(非负性),便于在输运方程中作为单一算子处理。

散射截面与散射分布函数共同决定了从所有入射态 $(E',\bOmega')$ 向给定出射态 $(E,\bOmega)$ 的“耦合强度”,其基本性质包括:对能角变量的可积性与非负性、对总截面的正确归一,以及在各向同性介质中的角向对称性。这些性质保证了散射算子在数学上具有良好的保正性和守恒结构,也为后续数值离散和近似模型设计提供了可靠的物理约束。


在许多辐射输运与医学物理问题中,若过程可视为稳态($\partial_t I = 0$)、并且能量变量经过适当简化,则上述一般输运方程可以退化为只含位置与角度的稳态 RTE。为避免符号混乱,本文在后续章节中主要采用简化形式讨论 RTE 的解算算子结构和数值性质,但在概念上始终可以从上述时空–能量–角度统一的高维输运方程出发来理解各类近似模型。

与前文讨论的玻尔兹曼方程类似,RTE 也是定义在高维相空间的偏微分–积分方程。在三维空间、二维角度变量($d=3$)的典型情形下,自变量维数达到七维,且散射积分项在角度空间引入了全局耦合,使得解析解一般不可得,只能依赖数值方法。现有 RTE 数值方法可以粗略分为两大类。一类是基于偏微分方程求解器的确定性方法:在空间上常采用菱形差分(diamond difference)、迎风格式和其他有限差分方法,或有限元、有限体积、间断 Galerkin 等离散手段;在角度上则使用 $P_n$ 方法、有限元或离散纵标(discrete ordinates, DOM)等,将角度积分用有限和近似。离散之后需要求解规模较大的线性代数系统,且由于散射积分的存在,系数矩阵往往并不十分稀疏,因而通常需要迭代法并配合加速技术。另一类是蒙特卡洛方法,它通过在相空间中随机采样粒子轨迹和散射事件,统计近似 RTE 解,具有几何处理灵活、易于并行等优点,但收敛速度遵循 $1/\sqrt{N}$ 规律,高精度计算往往需要大量样本,计算代价较高。

\section{国内外研究现状}

针对偏微分方程开发高效稳定的数值方法一直是一个具有挑战性而又重要的研究课题,在过去的十年里引起了很多学者的关注。从整体上看,现有工作大致可以分为两条技术路线:一条是基于网格的传统数值方法,另一条是近年来兴起的基于神经网络的无网格方法。

\subsection{传统数值方法}

在传统数值分析框架下,输运方程通常通过显式离散来求解。对于确定性方法,常见做法是将输运方程视作一类偏微分或积分–微分方程,采用有限差分法、有限体积法、有限元/离散 Galerkin 法等基于网格的离散技术\cite{ref3,ref4,ref5},在空间和时间上构造适当的网格与格式,对连续方程进行近似;对于随机方法,则常用蒙特卡洛模拟,通过采样大量粒子轨迹及其散射、吸收事件,在统计意义上重构输运过程。为了进一步降低计算量,工程中还广泛使用扩散近似、线性输运近似等简化模型,以舍弃部分角度或能量分辨率换取更高的计算效率。

这些传统数值方法在简单几何、系数相对均匀、维度较低的情况下具有较为成熟的理论基础和工程经验。然而,一旦面向输运方程的典型应用场景——例如辐射输运、线性玻尔兹曼输运等——若同时存在复杂几何边界、高维参数空间、强非线性或高度不均匀介质,以及多尺度耦合结构,传统方法便会面临明显的扩展瓶颈:一方面,基于网格的确定性方法需要在空间、角度乃至能量维度上同时离散,自由度数目呈指数级增长(“维数灾难”),极大推高了存储和计算成本;另一方面,蒙特卡洛方法虽然在物理建模上较为直接,但收敛速率缓慢、统计噪声较大,难以满足实时或多轮优化场景的需求。即便引入扩散近似等模型降阶技术,也往往以牺牲精度为代价,难以在高精度剂量评估或精细输运模拟中长期依赖。

综上所述,在高维、高复杂度输运问题中,传统数值方法在精度、效率与可扩展性之间存在难以调和的矛盾,这也为引入新型数值范式(如基于神经网络的求解框架)提供了直接动机。

\subsection{深度学习方法}

近年来,深度学习方法作为一种无网格的数值近似工具,在偏微分方程(PDE)求解方面展现出强大的潜力。其基本思想是利用(深层)神经网络表示 PDE 的解或解算子,通过最小化与 PDE 相关的损失函数来确定网络参数,从而在不显式构造网格的情况下,对高维复杂问题进行近似。与传统基于网格的数值方法相比,神经网络方法在处理高维输入和参数空间、复杂几何及不规则系数时具有更强的适应能力;一旦训练完成,同一网络可以在不同分辨率上高效评估;在适当引入物理约束的前提下,还可以在较少标注数据的条件下获得较高精度的近似解。

从研究范式上看,基于神经网络的 PDE 方法大体可以分为两类:一类侧重于学习单个 PDE 的解函数本身(解学习),另一类则侧重于学习 PDE 的解算子(算子学习)。前者通常将解函数直接参数化为神经网络,用网络替代有限基函数的线性组合,针对特定 PDE 建立模型;后者则试图通过神经算子在无限维函数空间之间学习映射 $G:X\to Y$,使得在参数、边界条件或源项变化时,预先训练好的模型仍具有一定的泛化能力。

在解学习方向,最具代表性的是物理信息神经网络(physics-informed neural networks, PINNs)及其变体。PINNs 通过在损失函数中同时 penalize PDE 残差和初始/边界条件残差,将物理方程直接嵌入网络训练过程,在一定程度上缓解了对大规模监督数据的依赖。随后,又出现了深度伽辽金方法(DGM)、深度 Ritz 方法、深度 BSDE 方法等一系列基于残差最小化或变分原理的深度求解框架,它们在流体力学、量子力学、金融数学等领域取得了大量成功应用。与此同时,研究者也逐步认识到常规解学习框架的局限性:例如,在对流占优方程、存在强非线性或剧烈梯度的场景下,PINNs 训练往往不稳定且容易陷入局部最优;对于复杂几何或多参数族问题,常规 PINNs 需要为每一组几何/参数配置单独训练或微调网络,难以在工业级设计空间中大规模扫参。针对这些问题,后续工作提出了诸如物理信息点网(PIPN)、Deep-TFC 和 X-TFC 等改进框架,通过引入几何感知网络结构、解析约束表达或特殊训练策略,在一定程度上提升了求解精度和训练效率,但整体上仍以“针对单个方程或固定几何”的解学习为主。

在算子学习方向,近年来涌现出一系列神经算子方法,试图直接在函数空间之间学习从输入函数(如系数场、初始条件、边界条件或源项)到解函数的映射。代表性工作包括深度算子网络(DeepONet)、傅里叶神经算子(Fourier Neural Operator, FNO)及其变体、图神经算子等。DeepONet 通过构造 “branch–trunk” 结构,对输入函数进行低维编码,并在目标空间上利用适当的基函数展开来近似解算子,已在高速边界层、高超声速多物理多尺度问题、电对流、分数阶算子、随机微分方程等场景中展示了良好的性能,并衍生出贝叶斯 DeepONet、多尺度 DeepONet、物理约束 DeepONet 等变体;FNO 则在频域中进行卷积运算,充分利用了 PDE 解的谱性质,具有较高的精度和推理效率,并在 U-FNO 等扩展模型中通过引入多尺度结构和噪声正则化进一步提升了泛化能力。与解学习相比,算子学习框架在理论上更适合处理“参数族 PDE”的问题,但也对训练数据的多样性、网络结构设计以及物理先验的嵌入提出了更高要求。

在上述背景下,研究者开始尝试将神经网络方法应用于辐射输运方程及相关输运模型。辐射输运方程是一类定义在高维相空间上的积分–微分方程,涉及复杂几何边界、高维散射核以及多尺度耦合过程,直接精确求解代价极高。已有部分工作基于 PINNs 或类似的解学习框架,针对特定几何形状和参数设置,求解辐射输运方程或与之相近的动力学方程。这类方法通常将辐射强度或粒子通量直接作为网络输出变量,通过最小化 RTE 残差以及边界/初始条件残差来训练网络,在一定程度上验证了 PINNs 在输运问题上的可行性。但由于仍然属于单方程解学习范式,其对边界条件、源项以及介质参数的泛化能力有限:当边界条件或物理参数发生显著变化时,往往需要重新训练或大幅微调网络,难以满足实际工程中多参数、高维扫参的需求。

另一方面,面向输运问题的算子学习研究也逐渐展开。一些工作尝试利用 DeepONet、FNO 等神经算子框架,对线性玻尔兹曼输运方程或简化的辐射输运模型的解算子进行近似学习,将介质参数、源项以及边界条件作为输入函数映射到通量或剂量分布。然而,现有工作在建模高维散射核方面仍存在明显不足:出于计算和建模的考虑,很多方法要么对散射核采用较强的各向同性或低阶展开近似,要么干脆忽略散射核的能角依赖,将其视为常数,从而削弱了模型对复杂散射机制的刻画能力;也有部分工作仅将部分边界条件或源项纳入网络输入,未能系统地处理边界条件、多束照射以及内部源项的统一表征。此外,多数算子网络架构针对特定方程和几何配置进行定制,训练超参数和网络结构需要大量经验调优,迁移到其他类型的输运问题或更复杂的几何结构时,往往需要重新设计和验证。

综上所述,尽管基于神经网络的 PDE 解学习和算子学习方法在诸多动力学系统中取得了令人瞩目的成果,并在辐射输运及相关输运方程的求解上展现出一定潜力,但现有工作在以下几个方面仍存在明显局限。其一,多数方法在几何形状、介质参数和边界条件变化时的泛化能力有限,往往需要针对每一组工况进行重新训练或大幅微调网络,难以直接支撑大规模参数扫描和个体化放疗计划等应用场景。其二,在高维散射核的表达和利用方面仍存在较强简化假设:许多算子学习工作出于计算和建模上的考虑,要么将散射核视为各向同性常数,要么仅保留低阶角向展开,从而削弱了对复杂能角耦合散射机制的刻画能力。其三,面向辐射输运方程这一类高度复杂的输运模型,现有典型神经算子架构(如 DeepONet 和 FNO)在直接应用时也暴露出结构性限制:一方面,DeepONet 的 branch–trunk 结构往往依赖预先给定的网格或采样点布局,对辐射场解中普遍存在的高频波动和多尺度局部特征刻画不足;另一方面,FNO 虽然在规则网格上的谱表示具有较高效率,但在输入参数场与输出解场处于不同类型空间(例如不同几何、边界条件或多通道散射描述)时,很难在统一的傅里叶域中构造有物理意义的变换,且对复杂高维数据进行全局傅里叶变换本身计算代价较高。上述因素共同导致现有神经算子框架难以“开箱即用”地作为辐射输运方程的通用解算算子,还需要在网络结构设计、边界条件与高维散射核的显式建模以及训练策略等方面进行进一步的可行性探索。正是这些不足,为本文发展面向辐射输运方程的神经算子方法、系统评估其在复杂输运场景下的适用性,并尝试提升模型的泛化与可扩展性提供了直接动机。

\section{本文工作与贡献}

为应对上述挑战,本文提出了一种面向辐射输运方程的神经算子框架 DeepRTE。DeepRTE 在设计上融合了预训练的注意力机制网络结构,并在网络架构中内嵌辐射输运过程的基本物理规律,从而在算子学习的同时尽可能保持对输运动力学的物理一致性。与通用的算子网络不同,DeepRTE 直接学习从边界条件、宏观截面参数以及散射核到稳态解场的映射,实现了端到端的解算算子近似,这一设计针对前文提到的“高维散射核难以表达”“边界条件与参数改变时泛化不足”等问题进行了一体化建模。

本文的主要工作与贡献可以概括为两个方面:

\begin{enumerate}
    \item 围绕辐射输运方程的结构特点,本文提出了具有物理约束和结构先验的神经算子框架 DeepRTE,主要包括:
          \begin{enumerate}
              \item \emph{端到端的输运算子学习}:将边界条件、宏观截面参数以及高维散射核统一视作输入函数,直接学习其到稳态解场之间的算子映射,在一个统一框架下同时处理边界变化、介质参数变化以及散射机制变化,从而缓解传统算子网络对高维散射核表达不足、边界条件更改时需重新训练等问题。
              \item \emph{物理结构显式嵌入}:针对辐射输运过程中的衰减与散射机理,在网络中显式构造衰减模块和散射模块,引入基于 Green 函数积分形式的半解析结构,使得整体算子在设计上保持线性结构和物理一致性,提高了模型在外推场景下的可靠性和可解释性。
              \item \emph{注意力机制与高维参数表征}:在网络架构中引入预训练的注意力机制,使模型能够自适应地关注不同空间位置、角度与参数区域的相关性,在不显著增加参数量的前提下提升对复杂散射核和高维参数空间的表征能力。
              \item \emph{小参数规模下的高精度与泛化}:数值实验表明,在参数规模显著小于多输入算子网络(multiple-input operators, MIO)的前提下,DeepRTE 仍能在辐射输运算子学习任务中取得更优或相当的精度与泛化性能,表明在物理约束和结构先验的指导下,小规模、物理信息驱动的神经算子可以在科学计算场景中优于大规模、纯数据驱动的黑箱网络。
              \item \emph{零样本泛化能力(zero-shot)}:借助上述算子化建模与物理结构嵌入,DeepRTE 在新的边界条件和源项配置下表现出良好的零样本泛化能力,即在不进行额外再训练的情况下,仍能对未见过的边界形式和源项组合给出合理预测,从而显著降低了在放疗剂量计算等场景下多工况扫参的计算成本。
          \end{enumerate}
          总体而言,在算法层面,DeepRTE 试图在“表达能力—物理一致性—计算效率”之间取得平衡:一方面通过注意力机制和端到端算子学习提升对复杂散射核和高维参数空间的表征能力,另一方面通过嵌入输运方程的物理结构和模块化设计控制模型复杂度、增强泛化与可解释性。

    \item 基于上述算法框架,本文在 JAX 生态系中自主实现了完整的DeepRTE代码库,突出高性能计算能力、函数式建模范式与多GPU并行支持,主要包括:
          \begin{enumerate}
              \item \emph{基于 JAX 的高性能实现}:系统性地采用 JAX 的函数式编程范式与自动微分机制,结合 \texttt{jit} 编译和向量化算子,对输运算子中的积分、卷积和注意力运算进行加速,使得在较大空间网格和参数空间下的训练与推理仍能保持可接受的计算开销。
              \item \emph{多 GPU 并行训练与推理}:在 JAX 的 XLA 编译和并行机制基础上,自主实现适配 DeepRTE 任务的多 GPU 并行策略,将批处理、空间网格和参数维度的划分与算子学习需求相结合,实现跨设备的数据并行和必要的通信调度,从而显著提升大规模训练与参数扫查实验的效率。
              \item \emph{完整的实验与配置管理}:构建包含 DeepRTE 各类模型实现、可扩展配置文件系统以及针对数据集生成、模型训练、验证与实验管理的配套工具的代码库,便于在不同几何、参数和散射配置下系统开展数值实验和消融研究。
              \item \emph{可复现的开源资源}:预训练模型和数据集已通过 Hugging Face 等平台对外公开,相关资源(包括模型权重、数据集与配置文件)可在 \href{https://huggingface.co/mazhengcn/deeprte}{mazhengcn/deeprte}\footnote{\url{https://huggingface.co/mazhengcn/deeprte}} 和 \href{https://huggingface.co/datasets/mazhengcn/rte-dataset}{mazhengcn/rte-dataset}\footnote{\url{https://huggingface.co/datasets/mazhengcn/rte-dataset}} 获取。本文中出现的所有算法、模型和数值结果均可以通过公开的代码仓库和上述资源完整复现,为后续在更复杂几何、多物理耦合和临床场景下的扩展研究提供可靠的工程基础。
          \end{enumerate}
\end{enumerate}


\section{行文结构}