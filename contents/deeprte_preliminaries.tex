% 预备知识:RTE 解算算子的解析结构
\chapter{预备知识}

本章回顾稳态辐射传输方程(Radiative Transfer Equation, RTE)及其解算算子的基本解析结构,并给出后文将频繁使用的符号与记号。相关结果既为我们构造深度神经网络算子提供理论依据,也帮助理解数值模型中的各个模块含义。

\section{符号与记号}

本文中涉及的主要符号含义如下所示:

\begin{longtable}[c]{c c p{0.72\textwidth}}
    \toprule
    记号                            & 部分                          & 含义说明                                                                                                         \\
    \midrule
    \endhead

    \midrule
    \endfoot

    \bottomrule
    \endlastfoot

    $d$                           & 第~\ref{sec:preliminaries}~节 & 空间维数                                                                                                         \\
    $D$                           & 第~\ref{sec:preliminaries}~节 & 空间区域,$D\subset\mathbb{R}^d$                                                                                  \\
    $\partial D$                  & 第~\ref{sec:preliminaries}~节 & 区域 $D$ 的边界                                                                                                   \\
    $\mathbb{S}^{d-1}$            & 第~\ref{sec:preliminaries}~节 & $\mathbb{R}^d$ 中的单位球面                                                                                        \\
    $S_{d-1}$                     & 第~\ref{sec:preliminaries}~节 & 单位球面 $\mathbb{S}^{d-1}$ 的表面积,$S_{d-1}=2\pi^{d/2}/\Gamma(d/2)$                                                \\
    $\br$                         & 第~\ref{sec:preliminaries}~节 & 位置向量                                                                                                         \\
    $\bOmega$                     & 第~\ref{sec:preliminaries}~节 & 方向单位向量                                                                                                       \\
    $\bm{v}$                      & 第~\ref{sec:preliminaries}~节 & 速度向量,$\bm{v}=\lvert \bm{v} \rvert \, \bOmega$                                                                \\
    $I$                           & 第~\ref{sec:preliminaries}~节 & 辐射强度(radiative intensity)                                                                                    \\
    $I_{-}$                       & 第~\ref{sec:preliminaries}~节 & 入射边界强度,在 $\Gamma_{-}$ 上给定                                                                                    \\
    $\mu_t$                       & 第~\ref{sec:preliminaries}~节 & 总截面(total cross section)                                                                                     \\
    $\mu_s$                       & 第~\ref{sec:preliminaries}~节 & 散射截面(scattering cross section)                                                                               \\
    $\mu_a$                       & 第~\ref{sec:preliminaries}~节 & 吸收截面,$\mu_a=\mu_t-\mu_s$                                                                                     \\
    $p$                           & 第~\ref{sec:preliminaries}~节 & 散射核或相函数(phase function)                                                                                      \\
    $\bn(\br)$                    & 第~\ref{sec:preliminaries}~节 & 在 $\br\in\partial D$ 处的外法向单位向量                                                                               \\
    $\Gamma_{\pm}$                & 第~\ref{sec:preliminaries}~节 & 入/出射边界集合:$\Gamma_{\pm}=\{(\br,\bOmega)\in\partial D\times \mathbb{S}^{d-1}\mid \mp \bn(\br)\cdot\bOmega<0\}$ \\
    $s$                           & 第~\ref{sec:preliminaries}~节 & 特征线上路径长度参数                                                                                                   \\
    $s_{-}(\br,\bOmega)$          & 第~\ref{sec:preliminaries}~节 & 从 $\br$ 沿 $-\bOmega$ 方向到达边界的距离                                                                               \\
    $\tau_{\br,\bOmega}(s_1,s_2)$ & 第~\ref{sec:preliminaries}~节 & 光学厚度:$\tau_{\br,\bOmega}(s_1,s_2):=\int_{s_1}^{s_2} \mu_t(\br-s\bOmega)\,\mathrm{d}s$                        \\
    $G$                           & 第~\ref{sec:preliminaries}~节 & RTE 的格林函数                                                                                                    \\
    $\mathcal{A}$                 & 第~\ref{sec:preliminaries}~节 & 解算算子,$\mathcal{A}[I_-]=I$                                                                                    \\
    $\mathcal{J}$                 & 第~\ref{sec:preliminaries}~节 & 边界衰减算子(attenuation operator)                                                                                 \\
    $\mathcal{L}$                 & 第~\ref{sec:preliminaries}~节 & 提升算子(lifting operator),沿特征线积分并包含 $\mu_s$                                                                     \\
    $\mathcal{S}$                 & 第~\ref{sec:preliminaries}~节 & 散射算子(scattering operator)                                                                                    \\
\end{longtable}

\section{RTE 及解算算子的解析结构}
\label{sec:preliminaries}

设空间区域为有界 Lipschitz 域 $D\subset\mathbb{R}^d$,其边界为 $\partial D$,外法向单位向量为 $\bn(\br)$。令
\begin{equation}
    \Gamma_{\pm} := \{ (\br,\bOmega)\mid \br\in\partial D,\; \bOmega\in\mathbb{S}^{d-1},\; \mp\bn(\br)\cdot\bOmega<0 \},
\end{equation}
即 $\Gamma_{-}$ 为入射边界,$\Gamma_{+}$ 为出射边界。辐射强度 $I(\br,\bOmega)$ 满足带入射边界条件的稳态辐射传输方程:
\begin{equation}\label{eq:rte-with-bc-cn}
    \begin{aligned}
        \bOmega\cdot\nabla I(\br,\bOmega) + \mu_t(\br) I(\br,\bOmega)
                                     & = \frac{\mu_s(\br)}{S_{d-1}} \int_{\mathbb{S}^{d-1}} p(\bOmega,\bOmega^*) I(\br,\bOmega^*)\,\mathrm{d}\bOmega^*, &  & \text{在 } D\times\mathbb{S}^{d-1}, \\
        I|_{\Gamma_{-}}(\br,\bOmega) & = I_{-}(\br,\bOmega),                                                                                            &  & \text{在 } \Gamma_{-},
    \end{aligned}
\end{equation}
其中 $I_{-}$ 为已知的入射边界数据,$S_{d-1}=2\pi^{d/2}/\Gamma(d/2)$ 为单位球面的面积。

系数函数包括总截面 $\mu_t$、散射截面 $\mu_s$ 以及散射核 $p$。通常定义吸收截面为 $\mu_a = \mu_t - \mu_s$。散射核 $p(\bOmega,\bOmega^*)$ 表示粒子从入射方向 $\bOmega^*$ 散射到方向 $\bOmega$ 的概率密度。我们假设散射核满足互易性,即
\begin{equation}
    p(\bOmega,\bOmega^*) = p(\bOmega^*,\bOmega),
\end{equation}
并满足归一化条件
\begin{equation}
    \frac{1}{S_{d-1}}\int_{\mathbb{S}^{d-1}} p(\bOmega,\bOmega^*) \, \mathrm{d}\bOmega^* = 1,
    \quad\text{几乎处处对 }\bOmega\in\mathbb{S}^{d-1}.
\end{equation}
在许多应用中,还会进一步假设散射核只依赖于散射角,即
\begin{equation}
    p(\bOmega,\bOmega^*) = \tilde p(\bOmega\cdot\bOmega^*),
\end{equation}
但本文的方法同样适用于一般的散射核。

\subsection{解的存在唯一性与解算算子}

在更一般的情形中,$\mu_t$ 和 $\mu_s$ 可以依赖于位置和速度 $\bm{v}=|\bm{v}|\bOmega$。在适当的“次临界”条件下,可以在 $L^1$、$L^\infty$ 以及一般 $L^p$ 空间中证明 RTE 的解的存在唯一性。这里我们只给出与本文数值设定最相关的结论:

\begin{theorem}[解在 $L^2$ 中的存在唯一性]
    假设区域 $D$ 有界,设
    \begin{equation}
        \mu_a = \mu_t - \mu_s \ge 0, \quad \mu_s \ge 0,
    \end{equation}
    且
    \begin{equation}
        \|\mu_a\|_{L^\infty(D)}<\infty, \quad \|\mu_s\|_{L^\infty(D)}<\infty,
    \end{equation}
    以及散射核满足归一化条件
    \begin{equation}
        \frac{1}{S_{d-1}}\int_{\mathbb{S}^{d-1}} p(\bOmega,\bOmega^*)\,\mathrm{d}\bOmega^*=1,\quad \text{几乎处处对 }\bOmega\in\mathbb{S}^{d-1}.
    \end{equation}
    则对任意 $I_{-}\in L^2(\partial D\times\mathbb{S}^{d-1})$,带边界条件的 RTE~\eqref{eq:rte-with-bc-cn} 在 $L^2(D\times\mathbb{S}^{d-1})$ 中存在唯一解 $I$,且存在常数 $C>0$(只依赖于 $\operatorname{diam}(D)$、$\|\mu_a\|_{L^\infty(D)}$ 及 $\|\mu_s\|_{L^\infty(D)}$),使得
    \begin{equation}
        \|\bOmega\cdot\nabla I\|_{L^2(D\times\mathbb{S}^{d-1})}
        + \|I\|_{L^2(D\times\mathbb{S}^{d-1})}
        \le C\,\|I_{-}\|_{L^2(\partial D\times\mathbb{S}^{d-1})}.
    \end{equation}
\end{theorem}

在上述假设下,我们可以将~\eqref{eq:rte-with-bc-cn} 的解表示为一个关于输入 $I_{-}$ 及系数函数 $(\mu_t,\mu_s,p)$ 的解算算子:
\begin{equation}
    \mathcal{A}:(I_{-};\mu_t,\mu_s,p)\longmapsto I,
\end{equation}
即在给定入射边界强度与介质参数的情况下,$\mathcal{A}$ 将其映射为区域内部的辐射强度场 $I$。需要特别指出的是,$\mathcal{A}$ 关于边界数据 $I_{-}$ 是线性的,但关于系数函数 $(\mu_t,\mu_s,p)$ 则是非线性的。

本工作的目标是构造一个高效、精确的数值方法来近似解算算子 $\mathcal{A}$,并在此基础上设计相应的深度神经网络结构。为此,我们需要更细致地分析 $\mathcal{A}$ 的迭代结构。

\subsection{固定点形式与算子分解}

为得到更便于数值实现与网络设计的表达形式,我们首先回顾 RTE 的常规积分形式推导。定义沿特征线的光学厚度
\begin{equation}\label{eq:optical-depth-cn}
    \tau_{\br,\bOmega}(s_1,s_2)
    := \int_{s_1}^{s_2} \mu_t(\br-s\bOmega)\,\mathrm{d}s,
\end{equation}
以及从点 $\br$ 沿方向 $\bOmega$ 反向追溯至边界的距离
\begin{equation}
    s_{-}(\br,\bOmega) := \inf\{ s\ge 0\mid \br-s\bOmega\in\partial D\}.
\end{equation}

在只考虑纯衰减、无散射的情形下,$I$ 满足
\begin{equation}
    (\bOmega\cdot\nabla+\mu_t) J_b = 0, \quad J_b|_{\Gamma_{-}}=I_{-},
\end{equation}
其解可写为
\begin{equation}\label{eq:attenuation-op-cn}
    J_b(\br,\bOmega)= \mathcal{J} I_{-}(\br,\bOmega)
    := \exp\bigl(-\tau_{\br,\bOmega}(0,s_{-}(\br,\bOmega))\bigr)
    I_{-}(\br-s_{-}(\br,\bOmega)\bOmega,\,\bOmega),
\end{equation}
其中算子 $\mathcal{J}$ 体现了沿特征线从边界到内部的指数衰减。

当右端源项为已知的散射源 $\mu_s I$ 时,满足
\begin{equation}
    (\bOmega\cdot\nabla+\mu_t) J = \mu_s I, \quad J|_{\Gamma_{-}}=0,
\end{equation}
其解可写为
\begin{equation}\label{eq:lifting-op-cn}
    J = \mathcal{L} I (\br,\bOmega)
    := \int_0^{s_{-}(\br,\bOmega)}
    \exp\bigl(-\tau_{\br,\bOmega}(0,s)\bigr)
    \mu_s(\br-s\bOmega)\, I(\br-s\bOmega,\bOmega)\,\mathrm{d}s,
\end{equation}
这一定义中的算子 $\mathcal{L}$ 可被理解为“提升算子”:它将局部的散射源沿特征线积分,得到在 $D$ 内部的贡献。

另一方面,散射算子定义为
\begin{equation}\label{eq:scattering-op-cn}
    \mathcal{S} I(\br,\bOmega)
    := \frac{1}{S_{d-1}}\int_{\mathbb{S}^{d-1}} p(\bOmega,\bOmega^*) I(\br,\bOmega^*)\,\mathrm{d}\bOmega^*,
\end{equation}
它描述了各个方向的辐射通过散射耦合到方向 $\bOmega$ 的过程。

在形式上将算子 $(\bOmega\cdot\nabla+\mu_t)$ 反演,可得到 RTE 的积分算子形式
\begin{equation}\label{eq:rte-op-form-cn}
    I = \mathcal{L}\mathcal{S}I + \mathcal{J} I_{-},
\end{equation}
这是一个典型的第二类 Fredholm 积分方程,其中 $\mathcal{L}\mathcal{S}I$ 表示散射引起的内部源贡献,$\mathcal{J}I_{-}$ 则反映边界入射的衰减传播。

在适当假设下,可以证明算子 $\mathcal{L}\mathcal{S}$ 在加权 $L^p$ 空间中是收缩算子,从而方程~\eqref{eq:rte-op-form-cn} 存在唯一不动点,并可由固定点迭代
\begin{equation}\label{eq:fixed-point-iteration-cn}
    I^{m+1} = \mathcal{L}\mathcal{S} I^{m} + \mathcal{J} I_{-}
\end{equation}
收敛得到。当取初值 $I^0 = \mathcal{J} I_{-}$ 时,该迭代等价于经典的源迭代(source iteration)方法。由谱半径估计可知,
\begin{equation}
    \rho(\mathcal{L}\mathcal{S}) < 1,
\end{equation}
从而解算算子 admit 下面的级数展开:
\begin{equation}\label{eq:soln-op-series-cn}
    \mathcal{A}[\mu_t,\mu_s,p] I_{-}
    = \sum_{n=0}^{\infty} (\mathcal{L}\mathcal{S})^n \, \mathcal{J} I_{-}.
\end{equation}
这说明 RTE 的解可以看作是“无数次散射”的叠加,每一次 $\mathcal{L}\mathcal{S}$ 的作用对应一次散射过程。这样的迭代结构与深度神经网络的层级结构在形式上高度相似,为构造高效的神经算子提供了重要启发。

\subsection{格林函数视角}

为了进一步理解解算算子的结构,可以引入 RTE 的格林函数 $G(\br,\bOmega,\br',\bOmega')$,形式上定义为下列问题的(分布意义下)解:
\begin{equation}\label{eq:rte-greens-function-cn}
    \begin{aligned}
        (\bOmega\cdot\nabla+\mu_t) G
         & = \mu_s \mathcal{S} G,                              &  & \text{在 } D\times\mathbb{S}^{d-1}, \\
        G|_{\Gamma_{-}}
         & = \delta_{\{\br'\}}(\br)\,\delta(\bOmega-\bOmega'), &  & \text{在 } \Gamma_{-},
    \end{aligned}
\end{equation}
其中 $(\br',\bOmega')\in\Gamma_{-}$ 视为参数,$\delta_{\{\br'\}}$ 表示在边界 $\partial D$ 上以 $\br'$ 为支撑的 Dirac 分布,$\delta(\bOmega-\bOmega')$ 为方向空间上的 Dirac 函数。此时,RTE 的解可以表示为
\begin{equation}\label{eq:rte-op-cn2}
    I(\br,\bOmega)
    = \int_{\Gamma_{-}} G(\br,\bOmega,\br',\bOmega')\, I_{-}(\br',\bOmega')\,\mathrm{d}\br'\,\mathrm{d}\bOmega'
    = \mathcal{A}[\mu_t,\mu_s,p] I_{-}(\br,\bOmega).
\end{equation}

根据前述固定点迭代结构,格林函数可以形式化地写为
\begin{equation}
    G(\br,\bOmega,\br',\bOmega')
    = \sum_{n=0}^{\infty} (\mathcal{L}\mathcal{S})^n
    \mathcal{J}\bigl(\delta_{\{\br'\}}(\br)\,\delta(\bOmega-\bOmega')\bigr),
\end{equation}
进而解算算子可写成
\begin{equation}
    \mathcal{A}[\mu_t,\mu_s,p] I_{-}(\br,\bOmega)
    = \int_{\Gamma_{-}} G(\br,\bOmega,\br',\bOmega')\, I_{-}(\br',\bOmega')\,\mathrm{d}\br'\,\mathrm{d}\bOmega'.
\end{equation}

从数值角度看,如果介质参数 $(\mu_t,\mu_s,p)$ 被视为固定,则一个直接的思路是用多层感知机(MLP)直接近似格林函数 $G$。然而在实际问题中,$\mu_t$、$\mu_s$ 往往是空间依赖、甚至带有间断的函数,而 $p$ 定义在 $\mathbb{S}^{d-1}\times\mathbb{S}^{d-1}$ 上,$G$ 对这些参数函数的依赖关系十分复杂。若希望训练出的算子能够对一整类 admissible 的 $(\mu_t,\mu_s,p)$ 泛化,则必须在网络结构中显式体现 RTE 解算算子的上述迭代与积分结构。后一章提出的 Deep RTE 算子正是基于这一思路构造的。
