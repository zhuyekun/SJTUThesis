\section{Experiments and results}\label{sec:experiments}
In this section, we present the numerical experiments and results to verify the effectiveness of DeepRTE.
In practice, the RTE~\eqref{eq:rte} is a $6$-dimensional equation in the position variable $\br\in\mathbb{R}^3$ and angular variable $\bOmega\in\sS^2$ corresponding to $d=3$.
It can be reduced to lower dimensional equations.
In the Cartesian coordinate system, let
\begin{equation}
	\bOmega=(c,s,\zeta), \quad c =
		{\left(1-\zeta^{2}\right)}^{\frac{1}{2}} \cos\theta, \quad s =
		{\left(1-\zeta^{2}\right)}^{\frac{1}{2}} \sin\theta, \quad \text{for }|\zeta| \leq 1.
\end{equation}
Suppose that $\mut, \mus$ only depend on $x$, $y$ and $I$ is uniform along the $z$-axis. The functions
\begin{equation}
	\tilde{I}(x, y,\zeta,\theta) = \frac{1}{2}[I(x,y,z,c,s,\zeta) + I(x,y,z,c,s,-\zeta)],
\end{equation}
and
\begin{equation}
	\tilde{p}(\zeta,\theta,\zeta^*, \theta^*) = \frac{1}{2}[p(c,s,\zeta, c^*,s^*,\zeta^*) + p(c,s,\zeta, c^*,s^*,-\zeta^*)],
\end{equation}
are independent of $z$ and even in $\zeta$. Thus, Eq.~\eqref{eq:rte-with-bc} reduces to the following equation:
\begin{equation}
	\left(c\partial_x \tilde{I}(x,y,\zeta,\theta)+s\partial_y
	\tilde{I}(x,y,\zeta,\theta)\right)+\mut \tilde{I}(x,y,\zeta,\theta)=\frac{\mus}{2\pi}\int_{0}^{1}
	\int_0^{2\pi} \tilde{p}(\zeta, \theta, \zeta^*,\theta^*) \tilde{I}(x,y,\zeta^*,\theta^*) \diff{\theta^*}\diff{\zeta^*},
\end{equation}
with inflow boundary condition:
\begin{equation}
	\tilde{I}(x,y,\bOmega)=\tilde{I}_{-}(x,y,\bOmega), \quad \text{for }
	\bOmega\cdot\bm{n}<0,\quad (x, y)\in\partial D.
\end{equation}

The phase function $p$ is essential in the radiative transport equation and typically depends only on the angle between the incoming and outgoing directions. A common model for anisotropic scattering is the normalized Henyey-Greenstein (H-G) phase function, given by
\begin{equation}
	p(\bOmega,\bOmega^*) = p(\bOmega\cdot\bOmega^*) = \frac{1-g^2}{{\Bigl(1+g^2-2g\,(cc^*+ss^*+\zeta\zeta^*)\Bigr)}^{\frac{3}{2}}}.
\end{equation}
The asymmetry parameter $g$ can be adjusted to control the relative amounts of forward and backward scattering in $p$: $g=0$ corresponds to isotropic scattering, and $g=1$ gives highly peaked forward scattering. Although we use the H-G function here, the algorithm can be easily adapted to other scattering models.

\subsection{Dataset construction}\label{sec:dataset-construction}

The datasets for training and evaluating are generated using conventional numerical methods, i.e., sweeping method~\cite{koch1991solution,zeyao2004parallel}.
The source code is written in MATLAB and Python and is available at \href{https://github.com/mazhengcn/rte-dataset}{rte-dataset}.

\paragraph{Computational domain and discretization}
We conduct our numerical experiments on a rectangular spatial domain $D = [0, 1]\times[0, 1]\subset \mathbb{R}^2$ and angular variable space $\mathbb{S}^{2}$.
The number of discrete mesh points in each direction is $40$, and these points are uniformly distributed.
% Grid points are randomly sampled from discrete grid points following a uniform distribution.

\paragraph{Angular quadrature points}
The angular quadrature points are the positive roots of the
standard Legendre polynomial of degree $2N$ on the interval $[-1,1]$.
Here we take $N=3$, the quadrature points and weights of the
quadrature set are presented in Table~\ref{tab:quadrature}.

\begin{table}[htbp]
	\centering
	\begin{tabular}{ccccc}
		\toprule
		$\zeta_i$ & $\theta_i$ & $c_i$     & $s_i$     & $4 w_i$   \\
		\midrule
		0.2386192 & $\pi$/12   & 0.9380233 & 0.2513426 & 0.1559713 \\
		0.2386192 & 3$\pi$/12  & 0.6866807 & 0.6866807 & 0.1559713 \\
		0.2386192 & 5$\pi$/12  & 0.2513426 & 0.9380233 & 0.1559713 \\
		0.6612094 & $\pi$/8    & 0.6930957 & 0.2870896 & 0.1803808 \\
		0.6612094 & 3$\pi$/8   & 0.2870896 & 0.6930957 & 0.1803808 \\
		0.9324695 & $\pi$/4    & 0.2554414 & 0.2554414 & 0.1713245 \\
		\bottomrule
	\end{tabular}
	\caption{Angular quadrature points and weights of the quadrature set.
		$\zeta_i$ and $\theta_i$ are the quadrature points of the velocity
		space, $c_i$ and $s_i$ are the corresponding cosine and sine
		values, and $w_i$ is the quadrature weight.}\label{tab:quadrature}
\end{table}

\paragraph{Cross section}
The cross section $\mu_t(x,y)$ and $\mu_s(x,y)$ are defined based on the subdomain $D_\mu = [0.4, 0.6] \times [0.4, 0.6]$. The detailed definitions are:
\begin{equation} \label{eq:coefficients_definition}
	\begin{aligned}
		\mu_t(x,y) & =
		\begin{cases}
			U_t, \quad \text{where } U_t \sim \mathcal{U}(5,7) & \text{if } (x,y) \in D_\mu    \\
			10                                                 & \text{if } (x,y) \notin D_\mu
		\end{cases} \\
		\mu_s(x,y) & =
		\begin{cases}
			U_s, \quad \text{where } U_s \sim \mathcal{U}(2,4) & \text{if } (x,y) \in D_\mu    \\
			5                                                  & \text{if } (x,y) \notin D_\mu
		\end{cases}
	\end{aligned}
\end{equation}

\paragraph{Training dataset boundary conditions}
The training dataset uses delta-function inflow boundary conditions, approximated by Gaussian functions:
% The smoothed delta function, $\delta_{x^{\prime}}^\sigma(x)$, used to construct these beams is defined as:
% \begin{equation} \label{eq:smoothed_delta_redefined}
%   \delta_{{x^{\prime}}}^\sigma(x)=\frac{1}{\sigma \sqrt{\pi}}
%   \exp\Bigl(-\frac{{(x-x^{\prime})}^{2}}{\sigma^{2}}\Bigr).
% \end{equation}
% The incoming boundary conditions for the training dataset are:
\begin{equation}\label{eq:bc-condition}
	\centering
	\left \{ % tex-fmt: skip
	\begin{aligned}
		 & I_-(x=0,y,c>0,s;x_l^{\prime}=0,
		y_l^{\prime},c_l^{\prime},s_l^{\prime})
		=\delta_{\{y_l^{\prime}\}}^{\sigma_{\br}}(y)\delta^{\sigma_{\bOmega}}(c - c_l')\delta^{\sigma_{\bOmega}}(s - s_l'),
		\\
		 & I_-(x=1,y,c<0,s;x_r^{\prime}=1,
		y_r^{\prime},c_r^{\prime},s_r^{\prime})
		=\delta_{\{y_r^{\prime}\}}^{\sigma_{\br}}(y)\delta^{\sigma_{\bOmega}}(c - c_r')\delta^{\sigma_{\bOmega}}(s - s_r'),
		\\
		 & I_-(x,y=0,c,s>0;x_b^{\prime},y_b^{\prime}=0, c_b^{\prime},
		s_b^{\prime})
		=\delta_{\{x_b^{\prime}\}}^{\sigma_{\br}}(x)\delta^{\sigma_{\bOmega}}(c - c_b')\delta^{\sigma_{\bOmega}}(s - s_b'),
		\\
		 & I_-(x,y=1,c,s<0;x_t^{\prime},y_t^{\prime}=1,c_t^{\prime},
		s_t^{\prime})
		=\delta_{\{x_t^{\prime}\}}^{\sigma_{\br}}(x)\delta^{\sigma_{\bOmega}}(c - c_t')\delta^{\sigma_{\bOmega}}(s - s_t'),
	\end{aligned}
	\right. % tex-fmt: skip
\end{equation}
where the variables $y_l'$ (left), $y_r'$ (right), $x_b'$ (bottom), and $x_t'$ (top) are uniformly sampled from discrete mesh points along the corresponding boundary.
The angular variables $c_l'$, $c_r'$, $c_b'$, $c_t'$, $s_l'$, $s_r'$, $s_b'$, and $s_t'$ are uniformly sampled from discrete velocity directions that satisfy the inflow boundary conditions. The parameters $\sigma_{\br}$ and $\sigma_{\bOmega}$ are drawn from uniform distributions $\sigma_{\br} \sim \mathcal{U}(0.005\sqrt{2},0.02\sqrt{2})$ and $\sigma_{\bOmega} \sim \mathcal{U}(0.005\sqrt{2},0.01\sqrt{2})$, respectively.

\paragraph{Dataset normalization}
In training process, the intensity $I$ was normalized to the range $[0, 1]$ using the formula $(I - I_{\text{min}}) / (I_{\text{max}} - I_{\text{min}})$, where $I_{\text{min}}$ and $I_{\text{max}}$ represent the minimum and maximum intensity values observed in the dataset, respectively. This normalization helps stabilize the training process and improve model convergence.

% \begin{itemize}
%     \item $y_l'$ and $y_r'$ are the y-coordinates for the center of beams on the left ($x_0=0$) and right ($x_0=1$) edges, respectively. Similarly, $x_b'$ and $x_t'$ are the x-coordinates for beams on the bottom ($y_0=0$) and top ($y_0=1$) edges. These source coordinates are uniformly sampled from discrete grid points along the respective boundary.
%     \item $c'$ and $s'$ (representing $c_0, s_0$ in the definition of $P$) define the central angular direction of the incident beam.
%     \item $\sigma_{\br}$ and $\sigma_{\bOmega}$ control the spatial and angular spread of the Gaussian beams. They are sampled from uniform distributions: $\sigma_{\br} \sim \mathcal{U}(0.005\sqrt{2},\,0.02\sqrt{2})$ and $\sigma_{\bOmega} \sim \mathcal{U}(0.005\sqrt{2},\,0.01\sqrt{2})$.
% \end{itemize}

\paragraph{Scattering regimes}
The model was trained and tested across three distinct scattering regimes, defined by the scattering asymmetry parameter $g \in [0,1]$ (where $g=0$ denotes isotropic scattering and $g=1$ denotes fully forward-peaked scattering). The regimes are as follows:
\begin{itemize}\label{li:regimes}
	\item \textbf{Near isotropy}: \( g \sim \mathcal{U}(0, 0.2) \), characterized by nearly isotropic scattering, where light or neutrons scatter in all directions equally.
	\item \textbf{Moderate anisotropy}: \( g \sim \mathcal{U}(0.4, 0.6) \), representing a middle ground where scattering is neither fully isotropic nor strongly directional.
	\item \textbf{Highly forward peaking}: \( g \sim \mathcal{U}(0.7, 0.9) \),  where particles (photons/neutrons) scatter predominantly forward, similar to a spotlight. This presents modeling challenges because: (1) particle paths become highly correlated; (2) their positions and scattering directions are closely linked; and (3) minor changes in boundary conditions can drastically affect the results.
\end{itemize}
Table~\ref{tab:dataset} summarizes the parameters used in the dataset construction.
\begin{table}[htbp]
	\centering
	\begin{tabular}{lp{0.3\textwidth}ll}
		\toprule
		Category                             & Parameters                                      & Symbol                   & Value/Range                                     \\ \midrule
		\multirow{2}{*}{Spatial domain}      & Domain                                          & $D$                      & ${[0,1]}^2$                                     \\
		                                     & Subdomain                                       & $D_\mu$                  & ${[0.4,0.6]}^2$                                 \\
		\midrule
		\multirow{4}{*}{Cross section}       & \multirow{2}{*}{Total}                          & \multirow{2}{*}{$\mut$}  & $\mathcal{U}(5,7)$ in $D_\mu$                   \\
		                                     &                                                 &                          & $10$ in $D\backslash D_\mu$                     \\
		                                     & \multirow{2}{*}{Scattering}                     & \multirow{2}{*}{$\mus$}  & $\mathcal{U}(2,4)$ in $D_\mu$                   \\
		                                     &                                                 &                          & $5$ in $D\backslash D_\mu$                      \\
		\midrule
		\multirow{2}{*}{Discretization}      & \# of mesh points                               & $N_{\text{mesh}}$        & 40                                              \\
		                                     & \# of angular quadrature points                 & $N_\text{quad}$          & 24                                              \\ \midrule
		\multirow{5}{*}{Boundary conditions} & Beam spatial center coordinates                 & $y_l', y_r', x_b', x_t'$ & Sampled from mesh points                        \\
		                                     & \multirow{2}{*}{Beam angular quadrature points} & $c_l', c_r', c_b', c_t'$ & \multirow{2}{*}{Sampled from quadrature points} \\
		                                     &                                                 & $s_l', s_r', s_b', s_t'$ &                                                 \\
		                                     & Beam spatial std dev                            & $\sigma_{\br}$           & $\sqrt{2}\,\mathcal{U}(0.005, 0.02)$            \\
		                                     & Beam angular std dev                            & $\sigma_{\bOmega}$       & $\sqrt{2}\,\mathcal{U}(0.005, 0.01)$            \\
		\midrule
		\multirow{3}{*}{Scattering}          & \multirow{3}{*}{Asymmetry parameter}            & \multirow{3}{*}{$g$}     & $\mathcal{U}(0,0.2)$                            \\
		                                     &                                                 &                          & $\mathcal{U}(0.4,0.6)$                          \\
		                                     &                                                 &                          & $\mathcal{U}(0.7,0.9)$                          \\
		\bottomrule
	\end{tabular}
	\caption{Summary of hyperparameters for dataset construction. This table details the computational domain, cross sections, discretization, boundary conditions, and other parameters used.}\label{tab:dataset}
\end{table}

\subsection{Training details}

\paragraph{Training configuration}

\begin{table}[htbp]
	\centering
	\begin{tabular}{ll}
		\toprule
		Hyperparameters          & Value                      \\ \midrule
		Optimizer                & Adam                       \\
		Learning rate schedule   & Cosine annealing           \\
		Initial learning rate    & $1 \times 10^{-3}$         \\
		Batch size               & 8                          \\
		Epochs                   & 5000                       \\
		\# of training data      & 800                        \\
		\# of validation data    & 200                        \\
		\# of collocation points & 128                        \\
		Hardware                 & 4$\times$ NVIDIA 4090 GPUs \\
		\bottomrule
	\end{tabular}
	\caption{Summary of model training hyperparameters. This table details the optimizer and training hyperparameters, including learning rate, dataset settings, and hardware information.}
	\label{tab:training_details}
\end{table}
The model was trained using the Adam optimizer with a cosine annealing learning rate schedule.
The training process involved 800 samples per scattering regime, and its performance was evaluated on a separate validation set of 200 samples per regime.
Training was conducted on four NVIDIA 4090 GPUs and proceeded for 5000 epochs.
The initial learning rate was set to $1 \times 10^{-3}$, the batch size was 8, and the number of collocation points used in the loss computation (see Eq.~\eqref{eq:collocation_points}) was 128.
Table~\ref{tab:training_details} summarizes the training hyperparameters.

\paragraph{Network architecture}
The neural network hyperparameters of DeepRTE are detailed in Table~\ref{tab:parameters}.

\begin{table}[htbp]
	\centering
	\begin{tabular}{lll}
		\toprule
		Module Name                  & Hyperparameters                          & Value \\ \midrule
		\multirow{7}{*}{Attenuation} & \texttt{num\_layer}
		$N_{\text{mlp}}$             & 4                                                \\
		                             & \texttt{hidden\_dim}  $d_{\text{mlp}}$   & $128$ \\
		                             & \texttt{output\_dim}  $d_{\text{model}}$ & $16$  \\
		                             & \texttt{num\_head}    $N_{\text{head}}$  & $2$   \\
		                             & \texttt{key\_dim}     $d_k$              & $32$  \\
		                             & \texttt{value\_dim}   $d_v$              & $32$  \\
		                             & \texttt{output\_dim}  $d_{\tau}$         & $2$   \\
		\midrule
		\multirow{2}{*}{Scattering}  & \texttt{num\_block}   $N_{\ell}$         & $2$   \\
		                             & \texttt{latent\_dim}  $d_{\text{model}}$ & $16$  \\
		\bottomrule
	\end{tabular}
	\caption{DeepRTE network hyperparameters. The network consists of two modules: the attenuation module and the scattering module. The hyperparameters include the number of layers, hidden dimensions, output dimensions, number of heads, key dimensions, value dimensions, and latent dimensions.}\label{tab:parameters}
\end{table}

\subsection{Results}
\subsubsection{Accuracy validation}
\label{sec:acc}
This section evaluates the model's accuracy and computational efficiency. Accuracy is measured using the mean squared error (MSE) and root mean squared percentage error (RMSPE) between predictions and ground truth. All experimental findings presented in this subsection are derived from a validation dataset with identical distribution to the training dataset.

\begin{table}[htbp]
	\centering
	\begin{tabular}{@{}cccccc@{}}
		\toprule
		Model                    & \# of parameters       & Scattering regime      & \(g\) range & MSE (\(\times 10^{-10}\)) & RMSPE (\%) \\ \midrule
		\multirow{3}{*}{DeepRTE} & \multirow{3}{*}{37954} & Near isotropy          & (0, 0.2)    & 5.630                     & 2.827      \\ \cmidrule(l){3-6}
		                         &                        & Moderate anisotropy    & (0.4, 0.6)  & 5.453                     & 2.759      \\ \cmidrule(l){3-6}
		                         &                        & Highly forward peaking & (0.7, 0.9)  & 7.223                     & 3.181      \\ \bottomrule
	\end{tabular}
	\caption{Accuracy validation of DeepRTE model across three distinct scattering regimes. All validation experiments were conducted on a dataset sharing the same distribution as the training dataset. The table demonstrates that the model achieves high accuracy with RMSPE consistently below 3.2\% across all regimes.}\label{tab:accuracy}
\end{table}

Across all regimes, DeepRTE achieved an MSE not exceeding \(7.3 \times 10^{-10}\) and an RMSPE below 3.2\%, indicating high accuracy. Specifically, in the near isotropy regime, the MSE was \(5.630 \times 10^{-10}\) with an RMSPE of 2.827\%, reflecting excellent performance in nearly isotropic conditions. For moderate anisotropy regime, the MSE was slightly lower at \(5.453 \times 10^{-10}\), with an RMSPE of 2.759\%, showing comparable accuracy. Notably, in the highly forward peaking regime, where modeling is typically more challenging due to the directional nature of photon transport, the model recorded an MSE of \(7.223 \times 10^{-10}\) and an RMSPE of 3.181\%, demonstrating robust performance despite the increased complexity.

Table~\ref{tab:accuracy-figs} provides visual comparisons of $\Phi$ predictions for representative cases across three scattering regimes. For each regime, we display: (1) 3D ground truth, (2) 2D ground truth projection, (3) model prediction, and (4) absolute error between prediction and ground truth.

The Table~\ref{tab:time_compare} presents a runtime comparison between DeepRTE and traditional methods, highlighting the efficiency gains achieved by our approach. The fast sweeping method was implemented in two variants: a CPU implementation (Python with NumPy and SciPy) measured on an Intel(R) Xeon(R) Silver 4410Y (12 cores), which requires $193.0$ s, and a highly optimized GPU implementation (JAX with data sharding across devices, multi-GPU parallelism, and \texttt{jax.jit} for JIT (Just-in-Time) compilation, see Alg.~\ref{alg:fast-sweeping-gpu}) measured on $4\times$ NVIDIA GeForce RTX 4090, which requires $16.7$ s ($11.6\times$ speedup over the CPU). By contrast, DeepRTE performs inference in $2.3$ s on $4\times$ RTX 4090, achieving a $ 7.3\times$ speedup over the fast sweeping method on GPU and an $83.9\times$ speedup relative to the CPU baseline.

These results show that DeepRTE effectively captures the physics of scattering across different parameter settings.
The model achieves low error rates even in complex forward scattering scenarios.
With only $37,954$ parameters, DeepRTE ensures computational efficiency.
These characteristics make it well-suited for large-scale simulations requiring fast processing.

\begin{table}[htbp]
	\centering
	\begin{tabular}{@{}lccc@{}}
		\toprule
		Method                     & Device                        & Count    & Time (s) \\ \midrule
		Fast sweeping method (CPU) & Intel(R) Xeon(R) Silver 4410Y & 12 cores & 193.0    \\
		Fast sweeping method (GPU) & NVIDIA GeForce RTX 4090       & 4 GPUs   & 16.7     \\
		\midrule
		DeepRTE (GPU)              & NVIDIA GeForce RTX 4090       & 4 GPUs   & 2.3      \\ \bottomrule
	\end{tabular}
	\caption{Runtime comparison. The conventional fast sweeping method requires 193.0 s on an Intel(R) Xeon(R) Silver 4410Y (12 cores) and 16.7 s on $4\times$ NVIDIA GeForce RTX 4090 GPUs ($11.6\times$ over CPU). By contrast, DeepRTE completes inference in 2.3 s on $4\times$ RTX 4090, achieving a $7.3\times$ speedup over the GPU fast sweeping method and an $83.9\times$ speedup relative to the CPU baseline.}
	\label{tab:time_compare}
\end{table}

\begin{algorithm}[!htbp]
	\caption{Multi-GPU Fast-Sweeping Method}\label{alg:fast-sweeping-gpu}
	\small
	$\Def \text{ FastSweeping}(\mu_t, \mu_s, \{c_k\}, \{s_k\}, \{w_{k'}\}, p, \Delta x, \Delta y, D=4, \text{tol}=10^{-8}, \text{max\_iter}=1000)$:
	\begin{algorithmic}[1]
		\Comment{Data sharding: partition angular flux and scattering kernel across $D$ devices}
		\State$M \gets n_v / D$
		\Comment{Shard angular flux and scattering kernel; replicate cross-sections}
		\ForAll{device $d \in \{1,\dots,D\}$}
		\State$I^{d,[\ell]}_{i, j, k} \gets \texttt{jax.zeros}(n_x, n_y, M)$ \hfill $I^{d,[\ell]}_{i,j,k} \in \mathbb{R}^{n_x \times n_y \times M}$
		\State$p^d_{k,k'} \gets p_{k,k'}$ for $k \in [(d-1)M, dM)$ and all $k'$
		\Comment{Replicate material cross-sections to all devices}
		\State$\mu_t^d, \mu_s^d \gets \mu_t, \mu_s$
		\EndFor
		\Comment{Source iteration with convergence check}
		\ForAll{iteration $\ell \in [0, \text{max\_iter})$}
		\Statex{\hspace{1.5em}\color{Brown}\small\textit{\# \;Step 1: All-gather angular fluxes from all devices (cross-device communication)}}
		\State$I_{i,j,k'}^{[\ell]} \gets \text{All-Gather}(\{I_{i,j, k}^{d,[\ell]}\}_{d=1}^D)$ \hfill $k' \in [0, n_v)$
		\ForAll{device $d$ \textbf{in parallel}}
		\Statex{\hspace{1.5em}\color{Brown}\small\textit{\# \;Step 2: Compute local scattering moments on each device (local computation, no communication)}}
		\State$S^{d,[\ell]}_{i,j,k} \gets \displaystyle\sum_{k'=0}^{n_v-1} w_{k'}\, p^d_{k,k'}\,I^{[\ell]}_{i,j,k'}$ \hfill $k \in [(d-1)M, dM)$
		\Statex{\hspace{1.5em}\color{Brown}\small\textit{\# \;Step 3: Perform local angular sweeps on each device (local computation, no communication)}}
		\ForAll{$i,j$ \textit{in sweep order}}
		\State$I^{d,[\ell+1]}_{i,j,k} \gets \dfrac{c_k\dfrac{I^{d,[\ell]}_{i-1,j,k}}{\Delta x} + s_k\dfrac{I^{d,[\ell]}_{i,j-1,k}}{\Delta y} + (\mu_s)_{i,j}\,S^{d,[\ell]}_{i,j,k}}{\dfrac{c_k}{\Delta x} + \dfrac{s_k}{\Delta y} + (\mu_t)_{i,j}}$ \hfill $k \in [(d-1)M, dM)$
		\EndFor
		\EndFor
		\Comment{Check convergence}
		\If{$\text{RelativeError}(I^{[\ell+1]}, I^{[\ell]}) < \text{tol}$}
		\State\textbf{break}
		\EndIf
		\EndFor
		\Comment{Gather all angular fluxes from devices}
		\State$I^{[\ell]} \gets \text{All-Gather}(\{I^{d,[\ell]}\}_{d=1}^D)$
		\Ret$I^{[\ell]}$
	\end{algorithmic}
\end{algorithm}

\begin{table}[htbp]
	\centering
	\begin{tabular}{>{\centering\arraybackslash}m{0.9cm}cccc}
		\toprule
		$g$                         & $\Phi_{\text{label}}$                                             & $\Phi_{\text{label}}$                                          & $\Phi_{\text{predict}}$                                      & $|\Phi_{\text{label}}- \Phi_{\text{predict}}|$                 \\
		\midrule
		\raisebox{5em}{$(0,0.2)$}   & \includegraphics[width=0.22\textwidth]{val_g0.1_phi_label_3d.pdf} & \includegraphics[width=0.20\textwidth]{val_g0.1_phi_label.pdf} & \includegraphics[width=0.20\textwidth]{val_g0.1_phi_pre.pdf} & \includegraphics[width=0.20\textwidth]{val_g0.1_phi_error.pdf} \\
		\raisebox{5em}{$(0.4,0.6)$} & \includegraphics[width=0.22\textwidth]{val_g0.5_phi_label_3d.pdf} & \includegraphics[width=0.20\textwidth]{val_g0.5_phi_label.pdf} & \includegraphics[width=0.20\textwidth]{val_g0.5_phi_pre.pdf} & \includegraphics[width=0.20\textwidth]{val_g0.5_phi_error.pdf} \\
		\raisebox{5em}{$(0.7,0.9)$} & \includegraphics[width=0.22\textwidth]{val_g0.8_phi_label_3d.pdf} & \includegraphics[width=0.20\textwidth]{val_g0.8_phi_label.pdf} & \includegraphics[width=0.20\textwidth]{val_g0.8_phi_pre.pdf} & \includegraphics[width=0.20\textwidth]{val_g0.8_phi_error.pdf} \\
		\bottomrule
	\end{tabular}\caption{Visual comparison of density $\Phi$ predictions across scattering regimes: ground truth (3D and 2D views), model predictions, and absolute errors for representative validation cases.}\label{tab:accuracy-figs}
\end{table}

\begin{figure}
	\centering
	\includegraphics[width=\textwidth]{train_loss.pdf}
	\caption{Loss curve. Obtained from training the model on a delta function training dataset, illustrating the convergence behavior over epochs.}\label{fig:loss}
\end{figure}

\subsubsection{Linearity validation}\label{sec:linearity_val}

Our DeepRTE model architecture is specifically designed to preserve the linearity property with respect to the boundary function $I_{-}$.
According to \eqref{eq:greens-integral-nn}, the neural network $G^{\text{NN}}$ used in DeepRTE represents the kernel (i.e., the Green's function) of the solution operator $\ANN$ in integral form. Therefore, the overall input-output mapping DeepRTE model remains linear with respect to $I_{-}$ at both continuous and discrete levels.

To verify this claim, we validate the linearity of DeepRTE through numerical experiments that test both additivity and homogeneity properties. All tests below are conducted with fixed cross sections $\mu_t,\mu_s$ and phase function $p$ with the following settings:
\begin{equation}
  \mu_t(x,y) =
  \begin{cases}
    6, & \text{if } (x,y) \in D_\mu \\
    10 & \text{if } (x,y) \notin D_\mu
  \end{cases},
  \quad
  \mu_s(x,y) =
  \begin{cases}
    3, & \text{if } (x,y) \in D_\mu \\
    5 & \text{if } (x,y) \notin D_\mu
  \end{cases},
  \quad
  g=0.5, \quad \text{where } D_\mu = [0.4, 0.6]^2.
\end{equation}
For simplicity, we write the DeepRTE operator as $\ANN$ instead of $\ANN[\mu_t,\mu_s,p]$. The two fundamental linearity properties are defined as follows:

\textbf{1. Additivity}: The DeepRTE predicted solution corresponding to the sum of two boundary conditions should equals the sum of individual solutions, i.e.,
\begin{equation}\label{eq:additivity-val}
  \ANN(I_{-}^{(1)} + I_{-}^{(2)}) = \ANN I_{-}^{(1)} + \ANN I_{-}^{(2)}.
\end{equation}
In this test, we set two independent boundary conditions $I_{-}^{(1)}$ and $I_{-}^{(2)}$ chosen from our validation dataset as follows, that is, the incoming boundary conditions are in the form of~\eqref{eq:bc-condition} with variance $\sigma_{\br}=0.01$ and $\sigma_{\bOmega}=0.0075$:
\begin{equation}
  I_{-}^{(1)}:
  \begin{cases}
    y_l'=0.3875, y_r'=0.6125, x_b'=0.6125, x_t'=0.3875, \\
    c_l'=0.6931, c_r'=-0.6931, c_b'=0.2871, c_t'=-0.2871, \\
    s_l'=-0.2871, s_r'=0.2871, s_b'=0.6931, s_t'=-0.6931.
  \end{cases}
\end{equation}
and
\begin{equation}
  I_{-}^{(2)}:
  \begin{cases}
    y_l'=0.1375, y_r'=0.8625, x_b'=0.8625, x_t'=0.1375, \\
    c_l'=0.2871, c_r'=-0.2871, c_b'=-0.6931, c_t'=0.6931, \\
    s_l'=0.6931, s_r'=-0.6931, s_b'=0.2871, s_t'=-0.2871.
  \end{cases}
\end{equation}
We evaluate the MSE error between the left and right sides of~\eqref{eq:additivity-val}, i.e.,
% \begin{equation}
%   \|\ANN (I_{-}^{(1)}{+}I_{-}^{(2)}) - ( \ANN I_{-}^{(1)} + \ANN I_{-}^{(2)})\|_2= 1.581\times 10^{-19}.
% \end{equation}
\begin{equation}
  \ell(\ANN (I_{-}^{(1)}{+}I_{-}^{(2)}),\ANN I_{-}^{(1)} + \ANN I_{-}^{(2)})=1.581\times 10^{-19}.
\end{equation}
This verifies that the two sides of~\eqref{eq:additivity-val} are nearly identical, thus confirming the additivity property of DeepRTE.

We also compute the MSE and RMSPE between our model predicted solution $\ANN (I_{-}^{(1)}{+}I_{-}^{(2)})$ and the ground truth solution $\A (I_{-}^{(1)}{+}I_{-}^{(2)})$ which is solved using numerical solver. The MSE is $1.358\times10^{-8}$ and the RMSPE is $2.83\%$ which shows that our model can accurately predict the solution for the combined boundary condition $I_{-}^{(1)}{+}I_{-}^{(2)}$. The Fig.~\ref{fig:additivity} visualizes the results of this additivity test using the density defined $\Phi$ in~\eqref{eq:density}.
\begin{figure}[htbp]
  \centering
  \begin{subfigure}[b]{0.30\textwidth}
    \centering
    \includegraphics[width=\textwidth]{figs/delta_sigma_a3-sigma_t6-g0_5-r5-v9-phi_label.pdf}
    \caption{Density $\Phi$ of $\ANN (I_{-}^{(1)} + I_{-}^{(2)})$.}
  \end{subfigure}
  \hfill
  \begin{subfigure}[b]{0.3\textwidth}
    \centering
    \includegraphics[width=\textwidth]{figs/delta_sigma_a3-sigma_t6-g0_5-r5-v9-phi_pre.pdf}
    \caption{Density $\Phi$ of $\ANN I_{-}^{(1)} + \ANN I_{-}^{(2)}$}
  \end{subfigure}
  \hfill
  \begin{subfigure}[b]{0.3\textwidth}
    \centering
    \includegraphics[width=\textwidth]{figs/delta_sigma_a3-sigma_t6-g0_5-r5-v9-phi_error.pdf}
    \caption{Absolute error}
  \end{subfigure}
  \caption{Additivity verification of DeepRTE with $g=0.5$, $\mus=3$ and $\mut=6$.}
  \label{fig:additivity}
\end{figure}

To be more general, we also validate the additivity property on 100 samples of validation dataset with identical distribution to the training dataset with $g=0.5$ mentioned in above section (see Sec.~\ref{sec:acc}). For every sample in dataset with boundary conditions $I_{-}^{(1)}$, we randomly choose another sample with the boundary conditions $I_{-}^{(2)}$, construct new boundary condition $I_{-}^{(1)}{+}I_{-}^{(2)}$ and then use DeepRTE to evaluate the intensity of this three cases, finally compare $\ANN (I_{-}^{(1)}{+}I_{-}^{(2)})$ and $( \ANN I_{-}^{(1)}{+}\ANN I_{-}^{(2)})$. The MSE and RMSPE between two sides of~\eqref{eq:additivity-val} are $1.421\times 10^{-19}$ and $9.20\times10^{-6}\%$ accordingly, see Table~\ref{tab:linearity}.

\textbf{2. Homogeneity}: The predicted solution corresponding to a scaled boundary condition is the scaled solution:
\begin{equation}\label{eq:homogeneity-val}
  \ANN(\alpha I_{-}) = \alpha \ANN I_{-}, \quad \text{for any } \alpha \in \mathbb{R}.
\end{equation}
Similar to the additivity test, one of the example boundary condition $I_{-}$ chosen from our validation dataset as follows,
\begin{equation}
  I_{-}:
  \begin{cases}
    y_l'=0.3875, y_r'=0.6125, x_b'=0.6125, x_t'=0.3875, \\
    c_l'=0.6931, c_r'=-0.6931, c_b'=0.2871, c_t'=-0.2871, \\
    s_l'=-0.2871, s_r'=0.2871, s_b'=0.6931, s_t'=-0.6931,
  \end{cases}
\end{equation}
with variance $\sigma_{\br}=0.01$, $\sigma_{\bOmega}=0.0075$ and
\begin{equation}
  \alpha=5.
\end{equation}
The MSE error between the left and right sides of~\eqref{eq:homogeneity-val} is:
% \begin{equation}
%   \|\ANN (\alpha I_{-}) - \alpha \ANN I_{-}\|_2= 5.179\times 10^{-13}.
% \end{equation}
\begin{equation}
  \ell(\ANN (\alpha I_{-}), \alpha \ANN I_{-}) = 5.179\times 10^{-13}.
\end{equation}
This verifies that the two sides of~\eqref{eq:homogeneity-val} are nearly identical, thus confirming the homogeneity property of
DeepRTE.

We also report the MSE and RMSPE between predicted $\A^\text{NN} (\alpha I^-)$ and ground truth solution solved by numerical solver $\A I^-$ is $5.842\times10^{-9}$ and $2.19\%$ accordingly. This shows that our model can accurately predict the solution for the scaled boundary condition $\alpha I^-$. The Fig.~\ref{fig:homogeneity} visualizes the results of this homogeneity test using the density $\Phi$ defined  in~\eqref{eq:density}
\begin{figure}[htbp]
  \centering
  \begin{subfigure}[b]{0.3\textwidth}
    \centering
    \includegraphics[width=\textwidth]{figs/delta-sigma_a3_sigma_t6-g0_5-r15-v3-phi_label.pdf}
    \caption{Density $\Phi$ of $\ANN (\alpha I_{-})$.}
  \end{subfigure}
  \hfill
  \begin{subfigure}[b]{0.3\textwidth}
    \centering
    \includegraphics[width=\textwidth]{figs/delta-sigma_a3_sigma_t6-g0_5-r15-v3-phi_pre.pdf}
    \caption{Density $\Phi$ of $\alpha \ANN I_{-}$}
  \end{subfigure}
  \hfill
  \begin{subfigure}[b]{0.3\textwidth}
    \centering
    \includegraphics[width=\textwidth]{figs/delta-sigma_a3_sigma_t6-g0_5-r15-v3-phi_error.pdf}
    \caption{Absolute error}
  \end{subfigure}
  \caption{Homogeneity verification with $g=0.5$, $\mus=3$, $\mut=6$ and $\alpha=5$.}\label{fig:homogeneity}
\end{figure}

To be more general, we also validate the homogeneity property on $100$ samples of validation dataset with identical distribution to the training dataset with $g=0.5$ mentioned in above section (see Sec.~\ref{sec:acc}). In homogeneity test, scale factor $\alpha \sim \mathcal{U}(0,5)$, we multiply this random scale factor with each origin boundary condition $I_-$, and compare the $\ANN (\alpha I^-)$ and $\alpha \ANN I^-$. The MSE and RMSPE between two sides of~\eqref{eq:homogeneity-val} are $2.804\times 10^{-13}$ and $1.01\times 10^{-5}\%$ accordingly, see Table~\ref{tab:linearity}.
\begin{table}[htbp]
  \centering
  \begin{tabular}{lcc}
    \toprule
    & MSE  & RMSPE($\%$)  \\
    \midrule
    Additivity & $1.421\times 10^{-19}$  & $9.20\times 10^{-6}$ \\
    Homogeneity & $2.804\times 10^{-13}$  & $1.01\times 10^{-5}$ \\
    \bottomrule
  \end{tabular}
  \caption{Additivity and homogeneity verification in $100$ samples of validation dataset with identical distribution to the training dataset with $g=0.5$ mentioned in above section (see Sec.~\ref{sec:acc}).}
  \label{tab:linearity}
\end{table}

\textbf{3. Linearity combination}:
The linearity of $\ANN$ established above implies that the generalization error for an arbitrary inflow $I_-$ can be decomposed exactly as in the proof of Theorem~\ref{thm:error-estimate} (see also Eq.~\ref{eq:error_estimate}). In this subsection we design a numerical experiment to empirically verify this decomposition and to assess the relative magnitudes of the constituent terms.

% Our goal is to choose an inflow boundary condition not explicitly present in the training set, which means, approximate it by the particle (delta-function) dataset described in Sec.~\ref{thm:bc-error-estimate}, and measure each term appearing on the right-hand side of the theoretical estimate in Theorem~\ref{thm:error-estimate}.

% To numerically compare the error, we introduce a high-resolution reference discrete solver $\A^{\text{ref}}$ (i.e. fast sweeping), If $\A^{\text{ref}}$ is a bounded linear operator, Theorem~\ref{thm:error-estimate} is also holds for $\A^{\text{ref}}$ and $\ANN$ (see the proof of Theorem~\ref{thm:error-estimate}).
% \begin{equation}\label{eq:triangle-A-Aref-ANN}
%     \| (\A - \ANN) I_- \|_{L^2} \leq \| (\A - \A^{\text{ref}}) I_- \|_{L^2} + \| (\A^{\text{ref}} - \ANN) I_- \|_{L^2}
% \end{equation}
% Consequently, once $\A^{\text{ref}}$ is convergent to $\A$ and the network accurately regresses $\A^{\text{ref}}$, the total error has a upper bound. If there exist $\varepsilon_{\text{ref}}(h)$ such that
% \begin{equation}
%     \| (\A - \A^{\text{ref}}) I_- \|_{L^2} \le \varepsilon_{\text{ref}},
% \end{equation}
% then \eqref{eq:triangle-A-Aref-ANN} immediately yields
% \begin{equation}
%     \| (\A - \ANN) I_- \|_{L^2} \le \varepsilon_{\text{ref}} + \| (\A^{\text{ref}} - \ANN) I_- \|_{L^2}.
% \end{equation}
% \begin{equation}
% \begin{aligned}
%           \| (\ANN - \A) I_- \|_{L^2} \; &\leq \varepsilon_{\text{ref}} + C \|I_- - I_{-,\sigma}^{N_{\text{bc}}}\|_{L^2(\Gamma_-)} + \sum_{i,j}|w_{ij}|\,\|(\ANN-\A^{\text{ref}})\delta_{\{\br_i\}}^{\sigma_{\br}}(\br) \delta^{\sigma_{\bOmega}}(\bOmega-\bOmega_j))\|_{L^2},
% \end{aligned}
% \end{equation}
% \begin{equation}
% \begin{aligned}
%           \| (\ANN - \A) I_- \|_{L^2} \; &\leq  C \|I_- - I_{-,\sigma}^{N_{\text{bc}}}\|_{L^2(\Gamma_-)} + \sum_{i,j}|w_{ij}|\,\|(\ANN-\A^{\text{ref}})\delta_{\{\br_i\}}^{\sigma_{\br}}(\br) \delta^{\sigma_{\bOmega}}(\bOmega-\bOmega_j))\|_{L^2},
% \end{aligned}
% \end{equation}
% where $C$ depends on the operator norms of $\A^{\text{ref}}$ and $\ANN$, see the proof of Theorem~\ref{thm:error-estimate}.

To numerically compare the error, for example, we consider a smooth inflow imposed only on the left boundary and zero elsewhere:
\begin{equation}
  \left \{ % tex-fmt: skip
  \begin{aligned}
    & I_-(x=0,y,c>0,s) = \sin(\pi y), \\
    & I_-(x=1,y,c<0,s) = 0,           \\
    & I_-(x,y=0,c,s>0) = 0,           \\
    & I_-(x,y=1,c,s<0) = 0.
  \end{aligned}
  \right. % tex-fmt: skip
\end{equation}

Following Eq.~\eqref{eq:particle-approx} and Eq.~\eqref{eq:delta_defination}, we only need to approximate the left boundary condition as follows:
\begin{equation}
  I_-(x=0,y,c>0,s) = \sin(\pi y) \approx \sum_{i=0}^{39} w_{i} \delta^{\sigma_{\br}}_{\{y_i\}}(y),
\end{equation}
where $\{y_i\}$ are the centers of uniform discrete cells, $g=0.5$, $\sigma_{\br} = 0.01$. And according to ~\cite{chertock2017practical} Section 2.1,
\begin{equation}
  w_i=\frac{1}{40}\sin(\pi y_i).
\end{equation}

To empirically verify the error estimate in Theorem~\ref{thm:error-estimate}, we use numerical solution as the reference solution $I^{\text{ref}}$ and measure the following errors (note all $L^2$ errors are computed by replacing integrals with quadrature over the discrete spatial--angular grid), starting from the regularized delta function approximation error:
\begin{equation}
  \|I_{-} - I^{40}_{-,0.01}\|_{L^2} \approx \frac{1}{160}\|I_{(x=0,y,c>0,s)} - \sum_{i=0}^{39} w_{i} \delta^{\sigma_{\br}}_{\{y_i\}}(y)\|_{\ell^2}=4.799\times 10^{-5},
\end{equation}
and then
\begin{equation}
  \|\ANN_{\theta^*}(I_{-} - I^{40}_{-,0.01})\|_{L^2}\approx 2.190\times 10^{-7}, \quad
  \|\A(I_{-} - I^{40}_{-,0.01})\|_{L^2} \approx 2.208\times 10^{-7},
\end{equation}
and finally the generalization error
\begin{equation}
  \|(\ANN_{\theta^*} - \mathcal{A}) I^{40}_{-,0.01}\|_{L^2}
  =\|\sum_{i=0}^{39}w_i(\ANN_{\theta^*} - \mathcal{A}) \delta^{\sigma_{\br}}_{\{y_i\}}(y)\|_{L^2} \leq \sum_{i=0}^{39}|w_i|\|(\ANN_{\theta^*} - \mathcal{A}) \delta^{\sigma_{\br}}_{\{y_i\}}(y)\|_{L^2} \approx 8.385\times 10^{-5},
\end{equation}
So the total error is bounded by $8.385\times 10^{-5}$.
We also calculate the direct inference error as
\begin{equation}
  \|(\ANN_{\theta^*} - \mathcal{A}) I_{-}\|_{L^2} \approx 1.888\times 10^{-5}.
\end{equation}
This is lower than our theoretical bound which verifies our results.

% From above two properties, we can conclude that for any linear combination of boundary conditions, the DeepRTE predicted solution satisfies:
% \begin{equation}
%   \ANN\left(\sum_{i=1}^{n} \alpha_i I_{-}^{(i)}\right) = \sum_{i=1}^{n} \alpha_i \ANN I_{-}^{(i)}, \quad \text{for any } \alpha       _i \in \mathbb{R}, n \in \mathbb{N}^+.
% \end{equation}
% This is a direct consequence of the additivity and homogeneity properties.
% This property is particularly useful in practical applications where complex boundary conditions can be decomposed into simpler components. By training the DeepRTE model on a set of fundamental boundary conditions, it can effectively generalize to more complex scenarios through linear combinations.

% We use numerical experiments to verify this. Only considering the left boundary as $\sin$ function:
% \begin{equation}
%   \left \{ % tex-fmt: skip
%   \begin{aligned}
%     & I_-(x=0,y,c>0,s) = \sin{\pi y},   \\
%     & I_-(x=1,y,c<0, s) = 0, \\
%     & I_-(x,y=0,c,s>0) =0,   \\
%     & I_-(x,y=1,c,s<0) =0,
%   \end{aligned}
%   \right. % tex-fmt: skip
% \end{equation}
% Substitute above $I_-$ to $f$, we can obtain that
% \begin{equation}
%   \begin{aligned}
%     I_-(\br, \bOmega) \approx I^{N_{\text{bc}}}_{-,\sigma} := \sum_{i,j}^{N_\text{bc}} w_{ij} I_-(\br, \bOmega)\frac{1}{\sigma_{\br} \sqrt{\pi}} \exp\left( -\frac{{(\br-\br_i)}^2}{\sigma_{\br} ^2}\right) \frac{1}{\sigma_{\bOmega} \sqrt{\pi}} \exp\left( -\frac{{(\bOmega-\bOmega_j)}^2}{\sigma_{\bOmega} ^2} \right),
%   \end{aligned}
% \end{equation}
% setting $\sigma_{\br} = 0.01$, $\sigma_{\bOmega} = 0.0075$ and $g=0.5$.
% For the right-hand side terms in Eq.~\ref{eq:error_estimate}, we obtain the following numerical results:
% \begin{equation}
%   \begin{aligned}
%     &\ell(I_-,I^{N_{\text{bc}}}_{-,\sigma}) = 1.474\times10^{-6}, \\
%     &\ell(\A^{\text{NN}}( I_-, I^{N_{\text{bc}}}_{-,\sigma})) = 1.841\times10^{-9}, \\
%     \sum^{N_{\text{bc}}}_{(\br_i,\bOmega_j)\in \Gamma_-}w_{ij}f_{ij}&\|(\A ^{\text{NN}}-\A) \delta^{\sigma}_{\{\br'\}}(\br)\delta^{\sigma}(\bOmega-\bOmega')\|_2 = 3.132\times10^{-10}, \\
%     &\| \A^{\text{ref}}( I_- - I^{N_{\text{bc}}}_{-,\sigma})\|_2 = 1.872\times10^{-9}.
%   \end{aligned}
% \end{equation}
% Here, $\A^{\text{ref}}$ denotes the reference numerical operator implemented via the sweeping method.

% Based on the boundedness discussed previously, we conclude that the error between the traditional sweeping-based operator and DeepRTE is well-controlled in this case.
% % \begin{table}[htbp]
% %   \centering
% %   \begin{tabular}{@{}lc@{}}
% %     \toprule
% %      & MSE \\
% %     \midrule
% %     % $\|\sin \br - \sum_{i,j} w_{ij}\sin(\br_i)\delta_{\br}\delta_{v}\|_2$ & $1.474\times10^{-6}$ \\
% %     $\|\A(\sin \br - \sum_{i,j} w_{ij}\sin(\br_i)\delta_{\br}\delta_{\bOmega})\|_2$ & $1.841\times10^{-9}$ \\
% %     $\|\sum_{i,j} w_{ij}\sin(\br_i)(\A-\A^{\text{NN}})\delta_{\br}\delta_{\bOmega}\|_2$ & $3.132\times10^{-10}$ \\
% %     \bottomrule
% %   \end{tabular}
% %   \caption{}
% %   \label{tab:error_decomp}
% % \end{table}

% % \begin{figure}[htbp]
% %   \centering
% %   % \begin{subfigure}[b]{0.24\textwidth}
% %   %   \centering
% %   %   \includegraphics[width=\textwidth]{figs/delta_sigma_a3-sigma_t6-g0_5-r5-v9-phi_label_3d.pdf}
% %   %   \caption{$\Phi_{\text{label}}$ (3D)}\label{fig:caseII_label_3d}
% %   % \end{subfigure}
% %   % \hfill
% %   \begin{subfigure}[b]{0.30\textwidth}
% %     \centering
% %     \includegraphics[width=\textwidth]{figs/test_sinx_label.pdf}
% %      \caption{Density of prediction which boundary condition is $\sin$}
% %   \end{subfigure}
% %   \hfill
% %   \begin{subfigure}[b]{0.3\textwidth}
% %     \centering
% %     \includegraphics[width=\textwidth]{figs/test_sinx_pre.pdf}
% %     \caption{A linear combination of predicted densities whose boundary condition is a delta function included in our training dataset.}\label{fig:caseII_predict}
% %   \end{subfigure}
% %   \hfill
% %   \begin{subfigure}[b]{0.3\textwidth}
% %     \centering
% %     \includegraphics[width=\textwidth]{figs/test_sinx_error.pdf}
% %     \caption{Absolute error}\label{fig:additivity}
% %   \end{subfigure}
% %   \caption{Evaluation dataset with scattering kernel coefficient $g=0.5$, $\mus=3$, $\mut=6$. MSE is $1.565\times10^{-11}$ and RMSPE is $8.916\times10^{-5}$}
% % \end{figure}


\subsubsection{Transfer learning and zero-shot performance}

In this section, we explore the DeepRTE framework's capabilities in transfer learning and Zero-Shot performance. Transfer learning enables a model to leverage knowledge from one task for improved performance on a related task, while Zero-Shot capability allows immediate application to entirely new conditions without retraining. To demonstrate these properties for the Radiative Transfer Equation (RTE)—specifically its ability to adapt to novel boundary conditions with minimal or no additional training—we train the model on datasets featuring delta function boundary conditions. The goal is to learn the RTE's fundamental physical principles. Subsequently, we evaluate the model on boundary conditions sampled from an entirely distinct data distribution (different from those in the training set), assessing its generalization capacity and Zero-Shot performance.
% —its ability to predict solutions for entirely new boundary conditions without further training.

To evaluate the model, we consider the following three specific boundary condition cases:
\begin{itemize}
	\item \textbf{Case I (Constant boundary conditions):} This case tests the model's generalization capability from the delta function boundary conditions used in training to a uniform source. It serves as a baseline for evaluating fundamental generalization.
	\item \textbf{Case II (Trigonometric boundary conditions):} This case introduces spatial changes using a sine function, assessing the model's ability to handle smooth, periodic, and spatially dependent boundary conditions.
	\item \textbf{Case III (Velocity dependent boundary conditions):} This case evaluates the model's performance with more complex, nonlinear boundary conditions that introduce directional dependence.
\end{itemize}

\paragraph{Case I. Constant boundary conditions}
The boundary conditions of the problem are
\begin{equation}
	\left \{
	\begin{aligned}
		 & I_-(x=0,y,c>0,s) =1,  \\
		 & I_-(x=1,y,c<0,s) = 0, \\
		 & I_-(x,y=0,c,s>0) =0,  \\
		 & I_-(x,y=1,c,s<0) =0,
	\end{aligned}
	\right.
\end{equation}
\begin{figure}[htbp]
	\centering
	\begin{subfigure}[b]{0.24\textwidth}
		\centering
		\includegraphics[width=\textwidth]{test_bc1_g0.1_label_3d.pdf}
		\caption{$\Phi_{\text{label}}$ (3D)}\label{fig:caseI_label_3d}
	\end{subfigure}
	\hfill
	\begin{subfigure}[b]{0.24\textwidth}
		\centering
		\includegraphics[width=\textwidth]{test_bc1_g0.1_label.pdf}
		\caption{$\Phi_{\text{label}}$}\label{fig:caseI_label}
	\end{subfigure}
	\hfill
	\begin{subfigure}[b]{0.24\textwidth}
		\centering
		\includegraphics[width=\textwidth]{test_bc1_g0.1_pre.pdf}
		\caption{$\Phi_{\text{predict}}$}\label{fig:caseI_predict}
	\end{subfigure}
	\hfill
	\begin{subfigure}[b]{0.24\textwidth}
		\centering
		\includegraphics[width=\textwidth]{test_bc1_g0.1_error.pdf}
		\caption{Absolute error}\label{fig:caseI_error}
	\end{subfigure}
	\caption{Evaluation dataset with scattering kernel coefficient
		$g\in (0,0.2)$. The RMSPE of the predicted solution is 2.573\%}
\end{figure}

\paragraph{Case II.\@ Trigonometric boundary conditions}
The boundary conditions of the problem are
\begin{equation}
	\left \{
	\begin{aligned}
		 & I_-(x=0,y,c>0,s) =a_L\sin{k_L y}+5,   \\
		 & I_-(x=1,y,c<0, s) = a_R\sin{k_R y}+5, \\
		 & I_-(x,y=0,c,s>0) =a_B\sin{k_B x}+5,   \\
		 & I_-(x,y=1,c,s<0) =a_T\sin{k_T x}+5,
	\end{aligned}
	\right.
\end{equation}

where
\begin{equation}
	a_L, a_R, a_B, a_T \sim \mathcal{U}(-5,5), \quad
	k_L, k_R, k_B, k_T \sim\mathcal{U}(-10, 10).
\end{equation}
\begin{figure}[htbp]
	\centering
	\begin{subfigure}[b]{0.24\textwidth}
		\centering
		\includegraphics[width=\textwidth]{test_sin_g0.1_label_3d.pdf}
		\caption{$\Phi_{\text{label}}$ (3D)}\label{fig:caseII_label_3d}
	\end{subfigure}
	\hfill
	\begin{subfigure}[b]{0.24\textwidth}
		\centering
		\includegraphics[width=\textwidth]{test_sin_g0.1_label.pdf}
		\caption{$\Phi_{\text{label}}$}\label{fig:caseII_label}
	\end{subfigure}
	\hfill
	\begin{subfigure}[b]{0.24\textwidth}
		\centering
		\includegraphics[width=\textwidth]{test_sin_g0.1_pre.pdf}
		\caption{$\Phi_{\text{predict}}$}\label{fig:caseII_predict}
	\end{subfigure}
	\hfill
	\begin{subfigure}[b]{0.24\textwidth}
		\centering
		\includegraphics[width=\textwidth]{test_sin_g0.1_error.pdf}
		\caption{Absolute error}\label{fig:caseII_error}
	\end{subfigure}
	\caption{Evaluation data set with scattering kernel coefficient
		$g\in (0,0.2)$. The RMSPE of the predict solution is 1.968\%}
\end{figure}

\paragraph{Case III.\@ Velocity dependent boundary conditions}The boundary conditions of the problem are
\begin{equation}
	\left \{
	\begin{aligned}
		 & I_-(x=0,y,c>0,s) =(a_{Lr}\sin{k_{Lr} y}+5)(a_{Lv}\sin{k_{Lv} c}+1)(a_{Lv}\sin{k_{Lv} s}+1),  \\
		 & I_-(x=1,y,c<0,s) = (a_{Rr}\sin{k_{Rr} y}+5)(a_{Rv}\sin{k_{Rv} c}+1)(a_{Rv}\sin{k_{Rv} s}+1),
		\\
		 & I_-(x,y=0,c,s>0) =(a_{Br}\sin{k_{Br} x}+5)(a_{Bv}\sin{k_{Bv} c}+1)(a_{Bv}\sin{k_{Bv} s}+1),
		\\
		 & I_-(x,y=1,c,s<0) =(a_{Tr}\sin{k_{Tr} x}+5)(a_{Tv}\sin{k_{Tv} c}+1)(a_{Tv}\sin{k_{Tv} s}+1),
	\end{aligned}
	\right.
\end{equation}

where
\begin{equation}
	\begin{aligned}
		a_{Lr}, a_{Rr}, a_{Br}, a_{Tr} & \sim \mathcal{U}(-5,5),   \\
		a_{Lv}, a_{Rv}, a_{Bv}, a_{Tv} & \sim \mathcal{U}(-1,1),   \\
		k_{Lr}, k_{Rr}, k_{Br}, k_{Tr} & \sim\mathcal{U}(-10, 10), \\
		k_{Lv}, k_{Rv}, k_{Bv}, k_{Tv} & \sim\mathcal{U}(-6, 6).
	\end{aligned}
\end{equation}
\begin{figure}[htbp]
	\centering
	\begin{subfigure}[b]{0.24\textwidth}
		\centering
		\includegraphics[width=\textwidth]{test_sin_rv_g0.1_label_3d.pdf}
		\caption{$\Phi_{\text{label}}$ (3D)}\label{fig:caseIII_label_3d}
	\end{subfigure}
	\hfill
	\begin{subfigure}[b]{0.24\textwidth}
		\centering
		\includegraphics[width=\textwidth]{test_sin_rv_g0.1_label.pdf}
		\caption{$\Phi_{\text{label}}$}\label{fig:caseIII_label}
	\end{subfigure}
	\hfill
	\begin{subfigure}[b]{0.24\textwidth}
		\centering
		\includegraphics[width=\textwidth]{test_sin_rv_g0.1_pre.pdf}
		\caption{$\Phi_{\text{predict}}$}\label{fig:caseIII_predict}
	\end{subfigure}
	\hfill
	\begin{subfigure}[b]{0.24\textwidth}
		\centering
		\includegraphics[width=\textwidth]{test_sin_rv_g0.1_error.pdf}
		\caption{Absolute error}\label{fig:caseIII_error}
	\end{subfigure}
	\caption{Evaluation data set with scattering kernel
		coefficient $g\in (0,0.2)$. The RMSPE of the predict
		solution is 1.908\%}
\end{figure}

\begin{table}[htbp]
	\centering
	\begin{tabular}{@{}cccc@{}}
		\toprule
		\textbf{}                 & Test dataset    & MSE                    & RMSPE (\%) \\ \midrule
		\multirow{3}{*}{Case I}   & $g\in(0,0.2)$   & $4.390 \times 10^{-6}$ & 1.833      \\
		                          & $g\in(0.4,0.6)$ & $5.184 \times 10^{-6}$ & 1.994      \\
		                          & $g\in(0.7,0.9)$ & $1.474 \times 10^{-5}$ & 3.193      \\ \midrule
		\multirow{3}{*}{Case II}  & $g\in(0,0.2)$   & $4.931 \times 10^{-4}$ & 1.653      \\
		                          & $g\in(0.4,0.6)$ & $5.798 \times 10^{-4}$ & 1.827      \\
		                          & $g\in(0.7,0.9)$ & $2.870 \times 10^{-3}$ & 3.572      \\ \midrule
		\multirow{3}{*}{Case III} & $g\in(0,0.2)$   & $1.065 \times 10^{-3}$ & 2.383      \\
		                          & $g\in(0.4,0.6)$ & $1.127 \times 10^{-3}$ & 2.452      \\
		                          & $g\in(0.7,0.9)$ & $1.853 \times 10^{-3}$ & 3.069      \\ \bottomrule
	\end{tabular}
	\caption{Transfer learning and Zero-Shot performance. The model, trained on delta function boundary conditions, is tested on constant (Case I), trigonometric (Case II), and combined trigonometric (Case III) boundary conditions for $g$ in ranges (0,0.2), (0.4,0.6), and (0.7,0.9). Performance is measured using mean squared error (MSE) and root mean square percentage error (RMSPE), demonstrating robust generalization with errors below 3.2\% for Case I, 3.6\% for Case II, and 3.1\% for Case III.}
\end{table}

\paragraph{Out-of-distribution possibility for the scattering kernel}
One last experiment we conducted is to test the performance of DeepRTE when the scattering kernel is outside the distribution of our pretraining dataset.
We test the performance as $g$ approaches $1$ and present the results for $g = 0.99$ under the following experimental setup:
\begin{equation}
	\mu_t(x,y) =
	\begin{cases}
		6, & \text{if } (x,y) \in D_\mu    \\
		10 & \text{if } (x,y) \notin D_\mu
	\end{cases},
	\quad
	\mu_s(x,y) =
	\begin{cases}
		3, & \text{if } (x,y) \in D_\mu    \\
		5  & \text{if } (x,y) \notin D_\mu
	\end{cases},
	\quad \text{where } D_\mu = [0.4, 0.6]^2,
\end{equation}
and the incoming boundary conditions are:
\begin{equation}
	\left \{
	\begin{aligned}
		 & I_-(x=0,y,c>0,s) =(5\sin{2\pi y}+5)(\sin{\pi c}+1)(\sin{\pi s}+1),  \\
		 & I_-(x=1,y,c<0,s) = (5\sin{2\pi y}+5)(\sin{\pi c}+1)(\sin{\pi s}+1),
		\\
		 & I_-(x,y=0,c,s>0) =(5\sin{2\pi x}+5)(\sin{\pi c}+1)(\sin{\pi s}+1),
		\\
		 & I_-(x,y=1,c,s<0) =(5\sin{2\pi x}+5)(\sin{\pi c}+1)(\sin{\pi s}+1).
	\end{aligned}
	\right.
\end{equation}

\begin{figure}[htbp]
	\centering
	\begin{subfigure}[t]{0.32\textwidth}
		\centering
		\includegraphics[width=\linewidth]{figs/test_g0.99_phi_label.pdf}
		\caption{$\Phi_{\text{label}}$}
		\label{fig:deeprte-g099-label}
	\end{subfigure}\hfill
	\begin{subfigure}[t]{0.32\textwidth}
		\centering
		\includegraphics[width=\linewidth]{figs/test_g0.99_phi_pre.pdf}
		\caption{$\Phi_{\text{predict}}$}
		\label{fig:deeprte-g099-predict}
	\end{subfigure}\hfill
	\begin{subfigure}[t]{0.32\textwidth}
		\centering
		\includegraphics[width=\linewidth]{figs/test_g0.99_phi_error.pdf}
		\caption{Absolute error}
		\label{fig:deeprte-g099-error}
	\end{subfigure}
	\caption{DeepRTE evaluation ($g=0.99$).}
	\label{fig:deeprte-g099}
\end{figure}

The figure above illustrates the reference solution $\Phi_{\text{label}}$, the predicted solution $\Phi_{\text{predict}}$, and the absolute error between them. Despite the challenging scenario with $g$ extremely close to 1, DeepRTE demonstrates remarkable accuracy. The quantitative metrics, including MSE of $0.00764$ and RMSPE of $4.257\%$, confirm DeepRTE's effectiveness and robustness in handling near-unity anisotropy. This performance highlights DeepRTE's capability to accurately solve the radiative transfer equation even under highly-peaked regime.


In conclusion, DeepRTE's transfer learning and zero-shot capabilities demonstrate its strength and effectiveness across different conditions for solving the radiative transport equation.
By applying the core physical principles learned during training, DeepRTE can accurately predict solutions for entirely new boundary conditions, even without additional training.
This ability to generalize and adapt to new problems highlights the potential of deep learning frameworks to advance scientific computing.


\subsection{Mesh dependence and cross-resolution performance}

An important characteristic of any numerical method for solving partial differential equations is its performance across various mesh resolutions. In this section, we test the mesh independence property of DeepRTE. This capability allows parameters learned on fine meshes to be effectively applied to coarser meshes without significant loss of accuracy.

\begin{table}[htbp]
	\centering
	\begin{tabular}{cccc}
		\toprule
		Test dataset & Mesh resolution & MSE                    & RMSPE (\%) \\
		\midrule
		\multirow{3}{*}{Evaluation Dataset}
		             & $40\times 40$   & $5.453\times 10^{-10}$ &
		$2.759$                                                              \\
		             & $20\times 20$   & $8.235 \times 10^{-9}$ &
		$10.006$                                                             \\
		             & $10\times 10$   & $9.476 \times 10^{-8}$ &
		$34.346$                                                             \\
		\midrule
		\multirow{3}{*}{Case I}
		             & $40\times 40$   & $4.390 \times 10^{-6}$ &
		$1.833$                                                              \\
		             & $20\times 20$   & $1.876 \times 10^{-5}$ &
		$3.758$                                                              \\
		             & $10\times 10$   & $1.243 \times 10^{-4}$ &
		$9.276$                                                              \\
		\midrule
		\multirow{3}{*}{Case II}
		             & $40\times 40$   & $4.931 \times 10^{-4}$ &
		$1.653$                                                              \\
		             & $20\times 20$   & $1.792 \times 10^{-2}$ &
		$9.952$                                                              \\
		             & $10\times 10$   & $3.687 \times 10^{-2}$ &
		$13.798$                                                             \\
		\midrule
		\multirow{3}{*}{Case III}
		             & $40\times 40$   & $1.065 \times 10^{-3}$ &
		$2.383$                                                              \\
		             & $20\times 20$   & $1.175 \times 10^{-2}$ &
		$8.132$                                                              \\
		             & $10\times 10$   & $4.511 \times 10^{-2}$ &
		$15.477$                                                             \\
		\bottomrule
	\end{tabular}
	\caption{Performance of DeepRTE across different resolutions and test sets. The Evaluation Dataset shows very high precision, with an MSE of $5.45\times10^{-10}$ on the $40\times 40$ grid. In contrast, Cases I-III show a consistent increase in error on coarser grids. This increase is mainly because of the limitations of numerical integration. With the second-order midpoint rule, halving the grid resolution should theoretically make the integration error four times larger, which matches our findings.}\label{tab:mesh_independence_multiple_testsets}
\end{table}

For the evaluation dataset and Case I, the root mean square percentage error rises from $1.8$--$2.8\%$ on the $40\times 40$ grid to $3.8$--$10\%$ on the $20\times 20$ grid. Cases II and III exhibit a larger but still manageable increase, with errors growing from $1.7$--$2.4\%$ to $8$--$10\%$ on the coarser grid.

This fourfold error increase is expected due to the theoretical limitations of the second-order midpoint rule used for the outermost numerical integration. This represents a key finding: DeepRTE has successfully learned the fundamental solution for problems with delta-function inflow boundary conditions, and the observed errors primarily originate from the numerical integration method rather than the neural network's approximation of the solution. Consequently, accuracy can be improved in the future by adopting more advanced numerical integration methods in place of the current midpoint approach. We plan to explore this further.

DeepRTE's ability to maintain good accuracy across different mesh resolutions suggests that the network is learning the basic physical principles of the radiative transport equation, not just fitting to a specific grid setup. This mesh independence has several important benefits:
\begin{enumerate}
	\item Computational efficiency: Users can train the model on
	      fine meshes for high accuracy, but use it on coarser
	      meshes for faster calculations when lower accuracy is acceptable.
	\item Flexibility: The same trained model can be used for
	      problems with different mesh needs without retraining.
\end{enumerate}
