\paragraph{Out-of-distribution possibility for the scattering kernel}
One last experiment we conducted is to test the performance of DeepRTE when the scattering kernel is outside the distribution of our pretraining dataset.
We test the performance as $g$ approaches $1$ and present the results for $g = 0.99$ under the following experimental setup:
\begin{equation}
	\mu_t(x,y) =
	\begin{cases}
		6, & \text{if } (x,y) \in D_\mu    \\
		10 & \text{if } (x,y) \notin D_\mu
	\end{cases},
	\quad
	\mu_s(x,y) =
	\begin{cases}
		3, & \text{if } (x,y) \in D_\mu    \\
		5  & \text{if } (x,y) \notin D_\mu
	\end{cases},
	\quad \text{where } D_\mu = [0.4, 0.6]^2,
\end{equation}
and the incoming boundary conditions are:
\begin{equation}
	\left \{
	\begin{aligned}
		 & I_-(x=0,y,c>0,s) =(5\sin{2\pi y}+5)(\sin{\pi c}+1)(\sin{\pi s}+1),  \\
		 & I_-(x=1,y,c<0,s) = (5\sin{2\pi y}+5)(\sin{\pi c}+1)(\sin{\pi s}+1),
		\\
		 & I_-(x,y=0,c,s>0) =(5\sin{2\pi x}+5)(\sin{\pi c}+1)(\sin{\pi s}+1),
		\\
		 & I_-(x,y=1,c,s<0) =(5\sin{2\pi x}+5)(\sin{\pi c}+1)(\sin{\pi s}+1).
	\end{aligned}
	\right.
\end{equation}

\begin{figure}[htbp]
	\centering
	\begin{subfigure}[t]{0.32\textwidth}
		\centering
		\includegraphics[width=\linewidth]{figs/test_g0.99_phi_label.pdf}
		\caption{$\Phi_{\text{label}}$}
		\label{fig:deeprte-g099-label}
	\end{subfigure}\hfill
	\begin{subfigure}[t]{0.32\textwidth}
		\centering
		\includegraphics[width=\linewidth]{figs/test_g0.99_phi_pre.pdf}
		\caption{$\Phi_{\text{predict}}$}
		\label{fig:deeprte-g099-predict}
	\end{subfigure}\hfill
	\begin{subfigure}[t]{0.32\textwidth}
		\centering
		\includegraphics[width=\linewidth]{figs/test_g0.99_phi_error.pdf}
		\caption{Absolute error}
		\label{fig:deeprte-g099-error}
	\end{subfigure}
	\caption{DeepRTE evaluation ($g=0.99$).}
	\label{fig:deeprte-g099}
\end{figure}

The figure above illustrates the reference solution $\Phi_{\text{label}}$, the predicted solution $\Phi_{\text{predict}}$, and the absolute error between them. Despite the challenging scenario with $g$ extremely close to 1, DeepRTE demonstrates remarkable accuracy. The quantitative metrics, including MSE of $0.00764$ and RMSPE of $4.257\%$, confirm DeepRTE's effectiveness and robustness in handling near-unity anisotropy. This performance highlights DeepRTE's capability to accurately solve the radiative transfer equation even under highly-peaked regime.
