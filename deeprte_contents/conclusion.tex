\section{Conclusions}\label{sec:conclusions}
In this paper, we introduce DeepRTE, a neural operator framework for solving the radiative transfer equation (RTE). By combining an attention-based architecture with operator learning strategy, DeepRTE directly maps input parameters such as boundary conditions, scattering coefficients, and scattering kernels to RTE solutions. Its key innovations include (1) parameter efficiency: achieving higher accuracy than large data-driven models with fewer parameters by encoding physical laws into the architecture; (2) zero-shot generalization: robustly predicting solutions for unseen boundary conditions without retraining; and (3) interpretability: maintaining linear, physically meaningful operations via physics-guided design.

The results of our experiments demonstrate the remarkable
transfer learning capabilities of DeepRTE.\@ Despite being
trained on a specific set of boundary conditions, the model
exhibits strong performance when applied to new boundary
conditions. The predicted solutions closely match the ground
truth with low error metrics.
This ability to generalize to unseen boundary conditions
highlights the effectiveness of DeepRTE in capturing the
underlying physical principles of the radiative transport equation.
Furthermore, the zero-shot performance of DeepRTE is
particularly impressive. Without any additional training or
fine-tuning, the model accurately predicts the solution of the
radiative transport equation for completely new boundary
conditions. 
% This zero-shot capability is a testament to the
% physics-informed architecture of DeepRTE, which enables it to
% leverage the learned physical principles to make accurate
% predictions in novel scenarios.

% In conclusion, the transfer learning and zero-shot capabilities
% of DeepRTE demonstrate its robustness and versatility in
% solving the radiative transport equation. By leveraging the
% underlying physical principles learned during training, DeepRTE
% can accurately predict solutions for new boundary conditions,
% even in the absence of additional training data. This ability
% to generalize and adapt to novel scenarios highlights the
% potential of physics-informed deep learning frameworks in
% advancing the field of scientific computing.
In conclusion, DeepRTE shows the power of combining deep learning with physical laws to solve complex scientific problems. Its ability to adapt to new conditions without extra training proves how physics-guided models can achieve accurate, reliable results even in unfamiliar scenarios. This approach not only improves efficiency for solving the radiative transfer equation but also offers a practical template for tackling other challenging equations in science and engineering.