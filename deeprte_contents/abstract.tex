\begin{abstract}
  % In this paper, we propose a novel neural network approach, termed DeepRTE, to address the steady-state Radiative Transfer Equation (RTE). The RTE is a differential-integral equation that governs the propagation of radiation through a participating medium, with applications spanning diverse domains such as neutron transport, atmospheric radiative transfer, heat transfer, and optical imaging. Our proposed DeepRTE framework leverages pre-trained attention-based neural networks to solve the RTE with high accuracy and computational efficiency. The efficacy of the proposed approach is substantiated through comprehensive numerical experiments.

In this paper, we propose a novel neural network approach, termed DeepRTE, to address the steady-state Radiative Transfer Equation (RTE).
The RTE is a differential-integral equation that governs the propagation of radiation through a participating medium, with applications spanning diverse domains such as neutron transport, atmospheric radiative transfer, heat transfer, and optical imaging.
Our DeepRTE framework demonstrates superior computational efficiency for solving the steady-state RTE, surpassing traditional methods and existing neural network approaches.
This efficiency is achieved by embedding physical information through derivation of the RTE and mathematically-informed network architecture.
Concurrently, DeepRTE achieves high accuracy with significantly fewer parameters, largely due to its incorporation of mechanisms such as multi-head attention.
Furthermore, DeepRTE is a mesh-free neural operator framework with inherent zero-shot capability. This is achieved by incorporating Green’s function theory and pre-training with delta-function inflow boundary conditions into both its architecture design and training data construction.
The efficacy of the proposed approach is substantiated through comprehensive numerical experiments.
\end{abstract}
% % REQUIRED
% \begin{keywords}
%     Deep Neural Networks, Neural Operator, Radiative Transfer
% Equation, Transformer.
% \end{keywords}

% % REQUIRED
% \begin{MSCcodes}
%     68Q25, 68R10, 68U05
% \end{MSCcodes}
